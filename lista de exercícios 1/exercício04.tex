\section*{Exercício 4}
\begin{definition}{Métrica induzida por uma norma}{métrica_norma}
    Seja \(\mathcal{E}\) um espaço vetorial dotado de uma norma \(\norm{\cdot} : \mathcal{E} \to [0, \infty)\). A aplicação
    \begin{align*}
        d : \mathcal{E} \times \mathcal{E} &\to [0, \infty)\\
                                     (x,y) &\mapsto \norm{x - y}
    \end{align*}
    é denominada métrica induzida pela norma \(\norm{\cdot}\).
\end{definition}

\begin{proposition}{Métrica induzida por uma norma}{métrica_norma}
    Seja \(\mathcal{E}\) um espaço vetorial sobre o corpo \(\mathbb{K}\). Uma métrica \(d : \mathcal{E} \times \mathcal{E} \to [0, \infty)\) é induzida por uma norma em \(\mathcal{E}\) se e somente se \(d\) satisfaz
    \begin{enumerate}[label=(\alph*)]
        \item invariância translacional: \(d(u + t, v + t) = d(u, v)\) para todo \(u,v,t \in \mathcal{E}\); e
        \item transformação de escala \(d(\alpha u , \alpha v) = \abs{\alpha}d(u, v)\) para todo \(u,v \in \mathcal{E}\) e \(\alpha \in \mathbb{K}\).
    \end{enumerate}
\end{proposition}
\begin{proof}
    Suponha que \(d\) é uma métrica induzida pela norma \(\norm{\cdot}\). Para todos \(u,v,t\in\mathcal{E}\) e \(\alpha \in \mathbb{K}\), temos
    \begin{equation*}
        d(u + t, v + t) = \norm{(u+t) - (v + t)} = \norm{u - v} = d(u,v)
    \end{equation*}
    e
    \begin{equation*}
        d(\alpha u, \alpha v) = \norm{\alpha (u - v)} = \abs{\alpha} \norm{u-v} = \abs{\alpha} d(u,v).
    \end{equation*}
    Isto é, se \(d\) é induzida por uma norma, então \(d\) satisfaz (a) e (b).

    Suponha agora que \(d\) satisfaz (a) e (b). Mostremos que a aplicação
    \begin{align*}
        \norm{\cdot} : \mathcal{E} &\to [0, \infty)\\
                                 v &\mapsto d(v, 0)
    \end{align*}
    é uma norma em \(\mathcal{E}\). Notemos que
    \begin{align*}
        v = 0 &\iff d(v, 0) = 0\\
              &\iff \norm{v} = 0,
    \end{align*}
    e
    \begin{equation*}
        \norm{\lambda u} = d(\lambda u, 0) = \abs{\lambda} d(u,0) = \abs{\lambda}\norm{u}
    \end{equation*}
    para todo \(\lambda \in \mathbb{K}\) e \(u \in \mathcal{E}\). Pela propriedade (b) segue que
    \begin{equation*}
        \norm{x + y} = d(x + y, 0) = d(x, -y),
    \end{equation*}
    portanto pela propriedade (a) e pela desigualdade triangular para \(d\), temos
    \begin{equation*}
        \norm{x+y} \leq d(x, 0) + d(0, y)
    \end{equation*}
    ou então \(\norm{x+y} \leq \norm{x} + \norm{y}\) para todo \(x,y\in\mathcal{E}\). Desse modo, \(\norm{\cdot}\) é uma norma em \(\mathcal{E}\). Ainda, temos
    \begin{equation*}
        d(u, v) = d(u-v, 0) = \norm{u - v},
    \end{equation*}
    portanto \(d\) é a métrica induzida pela norma \(\norm{\cdot}\). Isto é, se \(d\) satisfaz (a) e (b), então \(d\) é uma métrica induzida por uma norma.
\end{proof}
