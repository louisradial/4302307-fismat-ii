\section*{Exercício 6}
\begin{proposition}{\(\mathbb{Q}\) não é completo em relação à métrica usual}{cauchy_não_converge}
    A sequência \(\family{x_n}{n\in \mathbb{N}}\subset \mathbb{Q}\) definida por
    \begin{equation*}
        x_n = \sum_{k = 0}^n \frac{1}{k!}
    \end{equation*}
    é de Cauchy mas não converge a nenhum número racional em relação à métrica usual.
\end{proposition}
\begin{proof}
    Consideremos \(n,m \in \mathbb{N}\) com \(n > m\), então
    \begin{align*}
        \abs{x_n - x_m} &= \abs*{\sum_{k=0}^{n}\frac{1}{k!} - \sum_{j = 0}^m\frac{1}{j!}}\\
                        % &= \sum_{k=m+1}^{n} \frac{1}{k!}\\
                        &= \sum_{k=0}^{n-m-1} \frac{1}{(k+m+1)!}\\
                        &= \frac{1}{(m+1)!} \sum_{k=0}^{n-m-1} \frac{(m+1)!}{(k+m+1)!}\\
                        &= \frac{1}{(m+1)!} \left(1 + \frac{1}{m+2} + \frac{1}{(m+2)(m+3)} + \dots + \frac{(m+1)!}{n!}\right)\\
                        &\leq \frac{1}{(m+1)!}\left(1 + \frac{1}{m+2} + \frac{1}{(m+2)^2} + \dots + \frac{1}{(m+2)^{n-m-1}}\right)\\
                        &<\frac{1}{(m+1)!} \sum_{k = 0}^\infty (m+2)^{-k}.
    \end{align*}

    Pela \cref{prop:exercício_5}, temos
    \begin{equation*}
        \abs{x_n - x_m} < \frac{1}{(m+1)!}\frac{m + 2}{m + 1} < \frac{2}{(m+1)!},
    \end{equation*}
    para \(m > 0\). Assim, podemos tornar \(\abs{x_n - x_m}\) arbitrariamente pequeno ao escolher \(m\) suficientemente grande, isto é, a sequência é de Cauchy em relação à métrica usual.

    Suponhamos por contradição que a sequência converge a algum número racional \(e = \frac{p}{q}\), com \(p\) e \(q\) coprimos. Assim, dado \(\varepsilon > 0\), existe \(N > 0\) tal que
    \begin{equation*}
        n > N \implies \abs{e - x_n} < \varepsilon.
    \end{equation*}
    Pela desigualdade triangular, temos
    \begin{align*}
        \abs{e - x_m} &\leq \abs{e - x_n} + \abs{x_n - x_m}\\
                      &< \varepsilon + \frac{2}{(m+1)!}
    \end{align*}
    para \(m > 0\) e \(n > N\). Como \(\varepsilon\) é arbitrário, temos
    \begin{equation*}
        \abs{e - x_m} \leq \frac{2}{(m+1)!},
    \end{equation*}
    para \(m > 0\). Como a sequência é estritamente crescente, temos \(e > x_m\), logo
    \begin{equation*}
        x_m < e \leq x_m + \frac{2}{(m+1)!}.
    \end{equation*}
    Em particular, tomemos \(m = 2,\) então
    \begin{equation*}
        \frac52 < e \leq \frac{17}{6},
    \end{equation*}
    portanto \(2 < e < 3\), isto é, \(q \geq 2\), caso contrário \(e\) seria um inteiro entre inteiros consecutivos.

    Podemos tomar \(m = q\), de modo que
    \begin{align*}
        x_q < \frac{p}{q} \leq x_q + \frac{2}{(q+1)!} &\implies q!x_q < p(q-1)! \leq q!x_q + \frac{2}{(q+1)}\\
                                                      &\implies \sum_{k = 0}^{q} \frac{q!}{k!} < p(q-1)! < \sum_{k = 0}^{q} \frac{q!}{k!} + 1,
    \end{align*}
    já que
    \begin{equation*}
        q \geq 2 \implies \frac{2}{q+1} < 1.
    \end{equation*}
    Notemos que \(\frac{q!}{k!} \in \mathbb{N}\) para todo \(k \in \set{0, 1, \dots, q}\), isto é, \(p(q - 1)!\) é um número natural entre inteiros consecutivos. Essa contradição mostra que \(e \notin \mathbb{Q}.\) Desse modo, a sequência não é convergente nos racionais com a métrica usual.
\end{proof}
