\section*{Exercício 5}
\begin{proposition}{Sequência convergente nos números racionais}{exercício_5}
    Seja \(r > 1\) um número racional. A sequência \(\family{s_n}{n\in\mathbb{N}}\subset\mathbb{Q}\) definida por
    \begin{equation*}
        s_n = \sum_{k = 0}^{n} r^{-k}
    \end{equation*}
    é de Cauchy e converge a \(\frac{r}{r-1} \in \mathbb{Q}\) em relação à métrica usual.
\end{proposition}
\begin{proof}
    Para \(n \in \mathbb{N}\), consideremos a fatoração
    \begin{equation*}
        x^{n+1} - y^{n+1} = (x - y)\sum_{k = 0}^{n} x^{n - k} y^{k},
    \end{equation*}
    para quaisquer \(x,y \in \mathbb{R}\). Em particular, temos
    \begin{equation*}
        1 - \left(\frac1r\right)^{n+1} = \left(1 - \frac1r\right)\sum_{k=0}^{n} \left(\frac{1}{r}\right)^{k},
    \end{equation*}
    isto é,
    \begin{equation*}
        s_n = \frac{r - r^{-n}}{r - 1}.
    \end{equation*}

    Assim, consideremos \(n, m \in \mathbb{N}\) com \(n > m\). Temos
    \begin{equation*}
        \abs{s_n - s_m} = \frac{r^{-m} - r^{-n}}{r - 1} = \left(1 - r^{m-n}\right) \frac{r^{-m}}{r - 1}.
    \end{equation*}
    Como \(r > 1\) e \(n > m\) temos
    \begin{align*}
        0 < r^{-1} < 1 &\implies 0 < r^{m-n} < 1\\
                       &\implies -1 < -r^{m-n} < 0\\
                       &\implies 0 < 1 - r^{m-n} < 1,
    \end{align*}
    portanto podemos estimar que
    \begin{equation*}
        \abs{s_n - s_m} < \frac{r^{-m}}{r - 1}.
    \end{equation*}
    Tomando \(m\) suficientemente grande, podemos tornar \(\abs{s_n - s_m}\) suficientemente pequeno, isto é, a sequência é de Cauchy em relação à métrica usual.

    Notemos que para qualquer \(n \in \mathbb{N}\),
    \begin{equation*}
        \abs*{\frac{r}{r-1} - s_n} = \frac{r^{-n}}{r-1},
    \end{equation*}
    então pelo mesmo argumento, temos que a sequência converge a \(\frac{r}{r-1} \in \mathbb{Q}\) em relação à métrica usual.
\end{proof}
