\documentclass[12pt,a4paper]{article}

% Language and formatting
\usepackage{polyglossia}
\usepackage{csquotes} %fvextra to avoid warning?
\setmainlanguage[variant=brazilian]{portuguese}

% \usepackage[backend=biber, style=alphabetic, sorting=ynt]{biblatex}
% \addbibresource{bibliography.bib}

% \setmainfont{Palatino Linotype}
% \setmathfont{Palatino Linotype}
\usepackage[a4paper, margin=1.5cm]{geometry}
\usepackage{booktabs}

% % title header
% \usepackage{titleps}% http://ctan.org/pkg/titleps
% \makeatletter
% \newpagestyle{main}{% Define page style main
%     \sethead%
%     [\textbf\thepage][][\thechapter.\ \chaptertitle]% [<even-left>][<even-center>][<even-right>]
%     {\thesection.\ \sectiontitle}{}{\textbf\thepage}% {<odd-left>}{<odd-center>}{<odd-right>}
%     \setfoot{}{}{}% {<left>}{<center>}{<right>}
% }
% \pagestyle{main}% Use page style main

% Images
\usepackage{tikz,pgfplots}
\pgfplotsset{compat=1.18}
\usetikzlibrary{cd}
\usepackage{graphicx, caption, subcaption}
\usepackage{float}

% Math tools
\usepackage{amsfonts, mathtools, amssymb, amsmath, amsthm, enumitem, dsfont}
\usepackage{newpxtext, newpxmath}
% \numberwithin{equation}{section}
\usepackage[ISO]{diffcoeff}
\usepackage{tensor}
\usepackage{siunitx}

% Misc
\usepackage{luacolor}
\usepackage[breakable]{tcolorbox}

\difdef{fp}{}{
    outer-Ldelim = \left.,
    outer-Rdelim = \right|,
    sub-nudge=0 mu
}
\newcommand\todo[1][!]{{\color{Red} TODO: {#1}}}
\difdef{l}{i}{outer-Rdelim = \,, outer-Ldelim=}
\difdef{l}{dn}{style=d^}
\NewDocumentCommand\dli{}{\dl.i.}
\DeclareMathOperator\Riem{Riem}
\DeclareMathOperator\Orb{Orb}
\DeclareMathOperator\Stab{Stab}
\DeclareMathOperator\sgn{sgn}
\DeclareMathOperator\End{End}
\DeclareMathOperator\tr{tr}
\DeclareMathOperator\sech{sech}
\DeclareMathOperator\artanh{artanh}
\DeclareMathOperator\hor{hor}
\DeclareMathOperator\ver{ver}
\DeclarePairedDelimiter\ceil{\lceil}{\rceil}
\DeclarePairedDelimiter\floor{\lfloor}{\rfloor}
\DeclarePairedDelimiter\abs{\lvert}{\rvert}
\DeclarePairedDelimiter\norm{\lVert}{\rVert}
\DeclarePairedDelimiterX\inner[2]{\langle}{\rangle}{#1,\mathopen{}#2}
\DeclarePairedDelimiter\set{\{}{\}}

\newcommand\dsone{\ensuremath{\mathds1}}
\newenvironment{smallpmatrix}{\left(\begin{smallmatrix}}{\end{smallmatrix}\right)}
\newcommand\ract{\mathbin{\vartriangleleft}}
\newcommand\ractalt{\mathbin{\blacktriangleleft}}
\newcommand\lact{\mathbin{\vartriangleright}}
\newcommand\lactalt{\mathbin{\blacktriangleright}}
\newcommand\ad[1]{\operatorname{ad}_{#1}}
\newcommand\Ad[1]{\operatorname{Ad}_{#1}}
\newcommand\preim[2]{\operatorname{preim}_{#1}{\left(#2\right)}}
\newcommand\id[1]{\operatorname{id}_{#1}}
\newcommand\colorunderline[2]{{\color{#1}\underline{{\color{black}{#2}}}}}
\newcommand\Hom[2][]{\ensuremath{\operatorname{Hom}_{#1}{\left(#2\right)}}}
\newcommand\bundle[3]{\ensuremath{#1 \mathrel{\overset{#2}{\to}} #3}}
\newcommand\smooth[1]{\ensuremath{\mathcal{C}^\infty(#1)}}
\newcommand\sections[1]{\ensuremath{\Gamma\left(#1\right)}}
\newcommand\forms[2][]{\ensuremath{\Lambda^{#1}{\left({#2}\right)}}}
\newcommand\ffamily[3]{\ensuremath{\set*{#1}_{#2}^{#3}}}
\newcommand\family[2]{\ensuremath{\set*{#1}_{#2}}}
\newcommand\vetor[1]{\ensuremath{\boldsymbol{#1}}}
\newcommand\linear{\ensuremath{\mathrel{\tilde{\to}}}}
\newcommand\topology[1]{\ensuremath{\left(#1, \mathcal{O}_{#1}\right)}}
\newcommand\manifold[1]{\ensuremath{\left(#1, \mathcal{O}_{#1}, \mathscr{A}_{#1}\right)}}
\newcommand\restrict[2]{\ensuremath{\left.#1\right\rvert_{#2}}}
\newcommand\bfield[1]{\ensuremath{\diffp*{}{#1}}}
\newcommand\bvec[3][]{\ensuremath{\diffp*{#1}{#2}[#3]}}
\newcommand\bset[3]{\ensuremath{\set*{\diffp*{}{{#1}^1}[#3], \dots, \diffp*{}{{#1}^{#2}}[#3]}}}
\newcommand\pf[2][]{\ensuremath{{#2}_{\ast{#1}}}}
\newcommand\pb[2][]{\ensuremath{{#2}^{\ast}_{#1}}}

% catppuccin (latte)
\definecolor{Rosewater}{RGB}{220,138,120}
\definecolor{Flamingo}{RGB}{221,120,120}
\definecolor{Pink}{RGB}{234,118,203}
\definecolor{Mauve}{RGB}{136,57,239}
\definecolor{Red}{RGB}{210,15,57}
\definecolor{Maroon}{RGB}{230,69,83}
\definecolor{Peach}{RGB}{254,100,11}
\definecolor{Yellow}{RGB}{223,142,29}
\definecolor{Green}{RGB}{64,160,43}
\definecolor{Teal}{RGB}{23,146,153}
\definecolor{Sky}{RGB}{4,165,229}
\definecolor{Sapphire}{RGB}{32,159,181}
\definecolor{Blue}{RGB}{30,102,245}
\definecolor{Lavender}{RGB}{114,135,253}
\definecolor{Text}{RGB}{76,79,105}
\definecolor{Subtext1}{RGB}{92,95,119}
\definecolor{Subtext0}{RGB}{108,111,133}
\definecolor{Overlay2}{RGB}{124,127,147}
\definecolor{Overlay1}{RGB}{140,143,161}
\definecolor{Overlay0}{RGB}{156,160,176}
\definecolor{Surface2}{RGB}{172,176,190}
\definecolor{Surface1}{RGB}{188,192,204}
\definecolor{Surface0}{RGB}{204,208,218}
\definecolor{Base}{RGB}{239,241,245}
\definecolor{Mantle}{RGB}{230,233,239}
\definecolor{Crust}{RGB}{220,224,232}

% References
\usepackage{hyperref}
\usepackage[brazilian,capitalize,nameinlink,noabbrev]{cleveref}
% \captionsetup[figure]{hypcap=false}
\makeatletter
\hypersetup{
    pdftitle=\@title,
    pdfauthor=\@author,
    colorlinks=true,
    linkcolor=Mauve,
    citecolor=pink,
    filecolor=red,
    urlcolor=blue,
    bookmarksdepth=4
}
\makeatother

% tcolorbox environments
\tcbuselibrary{theorems}
% theorem
\newtcbtheorem[auto counter, crefname={Teorema}{Teoremas}]{theorem}{Teorema}%
{breakable,colback=Mauve!5,colframe=Mauve!95!black,fonttitle=\bfseries}{thm}

% definition
\newtcbtheorem[auto counter, crefname={Definição}{Definições}]{definition}{Definição}%
{breakable, colback=Pink!5,colframe=Pink!95!black,fonttitle=\bfseries}{def}

% proposition
\newtcbtheorem[auto counter, crefname={Proposição}{Proposições}]{proposition}{Proposição}%
{breakable,colback=Lavender!5,colframe=Lavender!95!black,fonttitle=\bfseries}{prop}

% lemma
\newtcbtheorem[auto counter, crefname={Lema}{Lemas}]{lemma}{Lema}%
{breakable,colback=Flamingo!5,colframe=Flamingo!95!black,fonttitle=\bfseries}{lem}

% exercício
\newtcbtheorem[auto counter, crefname={Exercício}{Exercícios}]{exercício}{Exercício}%
{breakable,colback=Sky!5,colframe=Sky!95!black,fonttitle=\bfseries}{ex}

\title{Física Matemática II\\Segunda Lista de Exercícios e Tarefas}
\author{Louis Bergamo Radial\\8992822}

\begin{document}
\maketitle
\section*{Exercício 1}
\begin{proposition}{Métrica trivial}{métrica_trivial}
    Seja \(X\) um conjunto não vazio, então \((X, d_\mathrm{t})\) é um espaço métrico, onde a função \(d_\mathrm{t} : X \times X \to \mathbb{R}\) é a métrica trivial, definida por
    \begin{equation*}
        d_\mathrm{t}(x,y) = \begin{cases}
            0, & \text{se }x=y,\\
            1, & \text{se }x\neq y,\\
        \end{cases}
    \end{equation*}
    para todo \(x,y \in X\).
\end{proposition}
\begin{proof}
    Pela definição da métrica trivial, temos
    \begin{equation*}
        d_\mathrm{t}(x,y) = 0 \iff x = y
    \end{equation*}
    para todo \(x,y \in X\). De mesma forma, pela simetria de relação de igualdade, temos
    \begin{equation*}
        d_\mathrm{t}(x,y) = d_\mathrm{t}(y,x).
    \end{equation*}
    Ainda, a imagem da função \(d_\mathrm{t}\) é contida na semirreta \([0,\infty)\),
    \begin{equation*}
        d_\mathrm{t}(X\times X) = \set*{0, 1} \subset [0, \infty).
    \end{equation*}
    Assim, resta mostrar que a métrica trivial satisfaz a desigualdade triangular.

    Consideremos \(x,y,z \in \mathbb{R}\), então segue que
    \begin{equation*}
        0 \leq d_\mathrm{t}(x,z) + d_\mathrm{t}(z,y) \leq 2,
    \end{equation*}
    com os únicos valores possíveis para a soma sendo \(\set{0,1,2}\). No caso em que \(x = y\), temos \(d_\mathrm{t}(x,y) = 0\), portanto
    \begin{equation*}
        d_\mathrm{t}(x,y) \leq d_\mathrm{t}(x,z) + d_\mathrm{t}(z,y)
    \end{equation*}
    é satisfeita de forma trivial. No caso em que \(x \neq y\), temos \(d_\mathrm{t}(x,y) = 1\), portanto pela transitividade da igualdade temos que
    \begin{equation*}
        1 \leq d_\mathrm{t}(x,z) + d_\mathrm{t}(z,y) \leq 2,
    \end{equation*}
    já que \(z\) não pode ser igual a tanto \(x\) quanto \(y\), de modo que
    \begin{equation*}
        d_\mathrm{t}(x,y) \leq d_\mathrm{t}(x,z) + d_\mathrm{t}(z,y).
    \end{equation*}
    Dessa forma, mostramos que a desigualdade triangular é satisfeita em todos os casos, portanto \((X,d_\mathrm{t})\) é um espaço métrico.
\end{proof}

\section*{Exercício 2}
\begin{corollary}\label{exercício2a}
    A função de Green do problema de Sturm \(u''(x) = f(x)\) onde \(u\) é definida no intervalo \(x \in [0,1]\) e satisfaz \(u'(0) = 0\) e \(u(1) = 0\) é dada por
    \begin{equation*}
        G(x,\xi) = \left\{\begin{aligned}
                \xi - 1, && 0 \leq x < \xi \leq 1\\
                x - 1, && 0 \leq \xi < x \leq 1
        \end{aligned}\right.
    \end{equation*}
\end{corollary}
\begin{proof}
    Identificando \(\alpha_1 = 0, \alpha_2 = 1, \beta_1 = 1,\) e \(\beta_2 = 0,\) o resultado segue da \cref{prop:exercício1}.
\end{proof}

\begin{proposition}{Autovalores e autofunções do problema de Sturm-Liouville \(u'' + \lambda u = 0\)}{exercício2b}
    Os autovalores e as autofunções normalizadas do problema de Sturm-Liouville
    \begin{equation*}
        u'' + \lambda u = 0,
    \end{equation*}
    onde \(u\) é definida no intervalo \(x \in [0,1]\) e satisfaz as condições de contorno \(u'(0) = 0\) e \(u(1) = 0\) são dados por
    \begin{equation*}
        \lambda_n = \left(n-\frac{1}{2}\right)^2\pi^2\quad\text{e}\quad
        u_n =\sqrt{2} \cos\left(\frac{2n-1}{2}\pi x\right),
    \end{equation*}
    para todo \(n \in \mathbb{N}\smallsetminus\set{0}.\)
\end{proposition}
\begin{proof}
    Notamos que a solução geral da equação diferencial do problema de Sturm-Liouville é
    \begin{equation*}
        u(x) = \begin{cases}
            A\cosh(\sqrt{-\lambda}x) + B\sinh(\sqrt{-\lambda}x), & \lambda < 0\\
            Ax + B, &\lambda = 0\\
            A\cos(\sqrt{\lambda}x)+ B\sin(\sqrt{\lambda}x),&\lambda > 0\\
        \end{cases}.
    \end{equation*}
    Notemos que pelas condições de contorno, a solução para \(\lambda = 0\) é a solução trivial, portanto podemos descartar este caso. Para \(\lambda < 0\), segue de \(u'(0) = 0\) que \(B = 0\), então como o cosseno hiperbólico tem imagem positiva, a única solução de \(u(1) = 0\) é \(A = 0\), isto é, este caso também leva apenas a soluções triviais. Nos resta apenas o caso \(\lambda > 0\), temos de \(u'(0) = 0\) que \(B = 0\), logo da outra condição de contorno obtemos
    \begin{equation*}
        u(1) = 0 \implies \sqrt{\lambda} = \left(n-\frac12\right)\pi,\quad\text{com}\quad n \in \mathbb{N}\smallsetminus\set{0}.
    \end{equation*}
    Deste modo, os autovalores do problema de Sturm-Liouville considerado são
    \begin{equation*}
        \lambda_n = \left(n-\frac12\right)^2\pi^2,
    \end{equation*}
    para \(n \in \mathbb{N}\smallsetminus\set{0}\).

    Para determinar as autofunções normalizadas, notamos que o produto interno para este problema de Sturm-Liouville coincide com o produto interno usual para o espaço de funções integráveis em \([0,1]\). Impondo que \(\inner{u_n}{u_n} = 1\), obtemos
    \begin{equation*}
        \int_{0}^{1} \dli{x}\abs{A}^2 \cos^2\left(\frac{2n-1}{2}\pi x\right) = 1 \implies \abs{A}^2 \int_{0}^{1}\dli{x} \frac{1 + \cos((2n-1)\pi x)}{2} = 1 \implies \abs{A} = \sqrt{2},
    \end{equation*}
    portanto as autofunções normalizadas do problema de Sturm-Liouville são
    \begin{equation*}
        u_n(x) = \sqrt{2} \cos\left(\frac{2n-1}{2}\pi x\right),
    \end{equation*}
    para \(n \in \mathbb{N}\smallsetminus\set{0}.\)
\end{proof}

\begin{corollary}\label{exercício2c}
    A função de Green para o problema de Sturm associado é dada por
    \begin{equation*}
        G(x, \xi) = - \frac{8}{\pi^2}\sum_{m = 0}^{\infty} \frac{\cos\left(\frac{2m + 1}{2}\pi x\right)\cos\left(\frac{2m + 1}{2}\pi \xi\right)}{(2m + 1)^2},
    \end{equation*}
    para todo \((x, \xi) \in [0,1]\times[0,1]\).
\end{corollary}
\begin{proof}
    Pela fórmula de Mercer, temos
    \begin{equation*}
        G(x,\xi) = - \sum_{n = 1}^{\infty} \frac{u_n(x) u_n(\xi)}{\lambda_n},
    \end{equation*}
    onde \(u_n\) é a autofunção normalizada associada ao autovalor \(\lambda_n\) do problema de Sturm-Liouville. Pela \cref{prop:exercício2b}, segue que
    \begin{align*}
        G(x, \xi) &= - \sum_{n = 1}^\infty \frac{\left[\sqrt{2}\cos\left(\frac{2n-1}{2}\pi x\right)\right]\left[\sqrt{2}\cos\left(\frac{2n-1}{2}\pi \xi\right)\right]}{\left(\frac{2n-1}{2}\right)^2\pi^2}\\
                  &= - \frac{8}{\pi^2} \sum_{n = 1}^\infty \frac{\cos\left(\frac{2n-1}{2}\pi x\right)\cos\left(\frac{2n-1}{2}\pi \xi\right)}{(2n-1)^2}.
    \end{align*}
    Fazendo a troca de variável de soma \(m = n - 1\), obtemos a expressão desejada.
\end{proof}
\begin{corollary}
    Vale a identidade
    \begin{equation*}
        \frac{\pi^2}{8} = \sum_{m = 0}^\infty \frac{1}{(2m+1)^2}.
    \end{equation*}
\end{corollary}
\begin{proof}
    Pelo \cref{exercício2a} e pela continuidade da função de Green, segue que \(G(0,0) = -1\). Desse modo, pelo \cref{exercício2c}, temos
    \begin{equation*}
        -\frac{8}{\pi^2} \sum_{m=0}^\infty \frac{1}{(2m+1)^2} = -1\implies
        \sum_{m=0}^\infty \frac{1}{(2m+1)^2} = \frac{\pi^2}{8},
    \end{equation*}
    como desejado.
\end{proof}

\begin{proposition}{Solução para o problema de Sturm \(u''(x) = (3 - x)e^x\)}{exercício2d}
    A solução do problema de Sturm
    \begin{equation*}
        u''(x) = (3-x)e^x
    \end{equation*}
    com condições de contorno \(u'(0) = 0\) e \(u(1) = 0\) é
    \begin{equation*}
        u(x) =
    \end{equation*}
\end{proposition}

\section*{Exercício 3}
\begin{definition}{Mapa logístico}{mapa_logístico}
    A aplicação
    \begin{align*}
        T_a : \mathbb{R} &\to \mathbb{R}\\
                       x &\mapsto ax(1 - x)
    \end{align*}
    é chamada de \emph{mapa logístico} ao parâmetro \(a \in \mathbb{R}\).
\end{definition}

\begin{proposition}{Pontos fixos do mapa logístico}{pontos_fixo_mapa_logístico}
    Os pontos fixos do mapa logístico \(T_a\) são dados por
    \begin{equation*}
        x^\alpha = 0 \quad\text{e}\quad x^\beta = \frac{a-1}{a},
    \end{equation*}
    onde \(x^\beta\) claramente só está definido para \(a \neq 0\). O ponto fixo \(x^\beta\) pertence a \([0,1]\) se e somente se \(a \geq 1\).
\end{proposition}
\begin{proof}
    A equação de ponto fixo para \(T_a\) é dada por
    \begin{equation*}
        x = ax(1 - x) \implies x(a - 1 - ax) = 0,
    \end{equation*}
    cujas soluções são justamente \(x^\alpha\) e \(x^\beta\), com \(x^\beta\) definido apenas para \(a \neq 0\).

    Notemos que \(x^\beta = 1 - \frac1a\), portanto para \(a \geq 1\), temos \(x^\beta \in [0,1) \subset [0,1]\), uma vez que \(x^\beta\) é crescente para \(a > 0\). Para \(x^\beta \in [0,1]\), temos
    \begin{align*}
        x^\beta \in [0,1] &\implies 1 - \frac{1}{a} \geq 0 \land 1 - \frac{1}{a} \leq 1\\
                          &\implies a \notin [0, 1) \land a \geq 1\\
                          &\implies a \geq 1,
    \end{align*}
    como desejado.
\end{proof}

\begin{proposition}{Restrição do mapa logístico}{restrição_logístico}
    Seja \(A = [0,1].\) Se \(a \in [0,4]\), a aplicação \(\restrict{T_a}{A} : A \to \mathbb{R}\) é um endomorfismo.
\end{proposition}
\begin{proof}
    Trivialmente, se \(a = 0\) então \(T_a(\mathbb{R}) = \set{0} \subset A\), logo \(\restrict{T_0}{A} : A \to A\). Assim, podemos supor \(a \neq 0\).

    Como \(T_a\) é uma função suave, pelo teorema de Weierstrass esta função admite valor máximo e mínimo no compacto \(A\). Como
    \begin{equation*}
        \diff{T_a}{x} = 0 \implies x = \frac12 \in A,
    \end{equation*}
    segue que os valores de máximo e mínimo de \(T_a\) em \(A\) só podem ocorrer em \(x = 0, x = 1\) e \(x = \frac12\), cujos valores são \(T_a(0) = T_a(1) = 0\) e \(T_a(\frac12) = \frac{a}4\). Desse modo, para \(a > 0\) temos que o máximo global de \(\restrict{T_a}{A}\) ocorre em \(x = \frac12\). Assim, segue que
    \begin{equation*}
        a \in (0, 4] \implies 0 \leq T_a(x) \leq \frac{a}{4} \leq 1
    \end{equation*}
    para todo \(x \in A\). Concluímos portanto que \(T_a(A) \subset A\) para \(a \in [0,4]\).
\end{proof}

\begin{proposition}{Pontos fixos da restrição do mapa logístico}{pontos_fixos_restrição}
    Para \(a \in [0,1],\) a aplicação \(\restrict{T_a}{A} : A \to A\) tem um único ponto fixo, a saber, \(x = 0\). Para \(a \in (1, 4],\) a aplicação apresenta dois pontos fixos distintos, \(x = 0\) e \(x = x^\beta\).
\end{proposition}
\begin{proof}
    Para \(a = 0\), a imagem da aplicação é o conjunto \(\set{0}\), portanto o único ponto fixo é \(x = 0\).

    Consideremos \(a \in (0, 4]\). Pela \cref{prop:pontos_fixo_mapa_logístico}, os pontos fixos de \(T_a : \mathbb{R} \to \mathbb{R}\) são \(x^\alpha = 0\) e \(x^\beta\), com \(x^\beta \in A\iff a \geq 1\). Desse modo, para \(a \in (0, 1),\) o único ponto fixo de \(\restrict{T_a}{A}: A \to A\) é \(x = 0\). Ainda, para \(a = 1, x^\beta = 0\), de modo que para \(a \in [0,1], \) temos o único ponto fixo \(x = 0\) em \(A\). Para \(a \in (1, 4],\) \(x^\beta \neq 0,\) de modo que \(\restrict{T_a}{A}\) apresente dois pontos fixos distintos em \(A\).
\end{proof}

\begin{proposition}{Condições para a restrição do mapa logístico ser uma contração}{contração_logístico}
    Para \(a \in [0,1),\) a aplicação \(\restrict{T_a}{A} : A \to A\) é uma contração. Para \(a \in (1, 4]\), a aplicação não é contrativa.
\end{proposition}
\begin{proof}

\end{proof}

\section*{Exercício 4}
\begin{proposition}{Autofunções do problema de Sturm-Liouville com \(u'' + u' + \lambda u = 0\)}{exercício4d}
    Os autovalores e as autofunções normalizadas do problema de Sturm-Liouville \(u'' + u' + \lambda u = 0\) com condições de contorno \(u(0) = 0\) e \(u'(1) = 0\) são
    \begin{equation*}
        \lambda_n =\left(n - \frac12\right)^2 \pi^2 + \frac14
        \quad\text{e}\quad
        u_n(x) =\sqrt{2} \exp\left(-\frac{1}{2}x\right)\cos\left(\frac{2n - 1}{2}\pi x\right)
    \end{equation*}
    para \(n \in \mathbb{N}\smallsetminus{0}\). Assim,
    \begin{equation*}
        G(x,\xi) = - 8 \sum_{n = 1}^\infty \frac{\exp\left(-\frac{x + \xi}{2}\right)\cos\left(\frac{2n - 1}{2} \pi x\right)\cos\left(\frac{2n - 1}{2} \pi \xi\right)}{\left(2n - 1\right)^2\pi^2 + 1},
    \end{equation*}
    para \((x,\xi) \in [0,1]\times[0,1]\), é a função de Green do problema de Sturm associado.
\end{proposition}
\begin{proof}
    Como já feito na \cref{prop:exercício3b}, temos as soluções gerais da equação diferencial, que dependem se \(\lambda\) é nulo ou se \(\lambda - \frac14\) é positivo, nulo, ou negativo. Para \(\lambda = 0\), temos \(u(x) = \alpha e^{-x} + \beta\), portanto de \(u'(1) = 0\), segue que \(\alpha = 0\), levando à solução trivial. Para \(\lambda = \frac14\), temos \(u(x) = (\alpha x + \beta)e^{-\frac12 x}\), logo de \(u(0) = 0\), segue que \(\beta = 0\), portanto
    \begin{equation*}
        u'(x) = \alpha \left(1 - \frac12 x\right)e^{-\frac12 x},
    \end{equation*}
    e temos de \(u'(1) = 0\) que \(\alpha = 0\). Para \(\lambda < \frac14\) e \(\lambda \neq 0\), temos a solução geral
    \begin{equation*}
        u(x) = \alpha \exp\left(\lambda_+ x\right) + \beta \exp\left(\lambda_-x\right),
    \end{equation*}
    onde \(\lambda_+ = \frac{-1 + \sqrt{1 - 4 \lambda}}{2}\) e \(\lambda_- = \frac{-1 - \sqrt{1 - 4 \lambda}}{2}\) são valores reais e distintos. Das condições de contorno, temos
    \begin{equation*}
        \begin{cases}
            \alpha + \beta = 0\\
            \lambda_+ \alpha e^{\lambda_+} + \lambda_- \beta e^{\lambda_-} = 0
        \end{cases} \implies
        \alpha (\lambda_+ e^{\lambda_+} - \lambda_- e^{\lambda_-}) = 0,
    \end{equation*}
    e \todo[mostrando que \(\lambda_+ e^{\lambda_+} - \lambda_- e^{\lambda_-}\) não se anula], segue que há apenas a solução trivial. Para \(\lambda > \frac14\), temos a solução geral
    \begin{equation*}
        u(x) = e^{-\frac12x}\left[\alpha \cos\left(x\sqrt{\lambda - \frac14}\right) + \beta \sin\left(x\sqrt{\lambda - \frac14}\right)\right],
    \end{equation*}
    portanto da condição de contorno \(u(0) = 0\), segue que \(\alpha = 0\), e da condição de contorno \(u'(1) = 0\), segue que as soluções não triviais devem satisfazer
    \begin{equation*}
        2 \sqrt{\lambda - \frac14}\cos\left(\sqrt{\lambda - \frac14}\right) - \sin\left(\sqrt{\lambda - \frac14}\right) = 0 \implies 2\sqrt{\lambda - \frac14} = \tan\left(\sqrt{\lambda - \frac14}\right).
    \end{equation*}
    Notemos que a equação transcendental obtida possui infinitas soluções para \(\lambda > \frac14\), uma vez que a imagem da tangente contém todos os números reais positivos e a imagem da raiz quadrada também. Seja \(\varphi_n\) a \(n\)-ésima solução positiva da equação transcendental \(2\varphi = \tan \varphi\), então os autovalores do problema de Sturm-Liouville são dados por
    \begin{equation*}
        \lambda_n = \varphi_n^2 + \frac14,
    \end{equation*}
    para \(n \in \mathbb{N} \smallsetminus\set{0}\).

    Notemos que o produto interno para este problema de Sturm-Liouville é dado por
    \begin{equation*}
        \inner{f}{g}_r = \int_{0}^{1} e^{t}\dli{t}\overline{f(t)}g(t)
    \end{equation*}
    para quaisquer funções integráveis \(f, g\) em \([0,1]\). Para determinar as autofunções, devemos impor que as soluções encontradas têm norma unitária em relação a este produto interno, isto é,
    \begin{align*}
        \inner{u_n}{u_n}_r = 1 &\implies \int_0^1 e^t \dli{t} \abs{\beta_n}^2 e^{-t}\cos^2\left(\varphi_n t\right) = 1\\
                               &\implies  \frac12\abs{\beta_n}^2 \left[\int_0^1 \dli{t} - \int_0^1 \dli{t} \cos\left(2\varphi_n t\right)\right] = 1
    \end{align*}
    % Deste modo,
    % \begin{equation*}
    %     u_n(x) = \sqrt{2} \exp\left(-\frac{1}{2}x\right)\cos\left(\frac{2n - 1}{2}\pi x\right)
    % \end{equation*}
    % são as autofunções para este problema de Sturm-Liouville, para \(n \in \mathbb{N}\smallsetminus\set{0}\).
    %
    % Pela fórmula de Mercer, a função de Green do problema associado é
    % \begin{align*}
    %     G(x,\xi) &= -\sum_{n = 1}^\infty \frac{u_n(x)u_n(\xi)}{\lambda_n}\\
    %              &= - 8 \sum_{n = 1}^\infty \frac{\exp\left(-\frac{x + \xi}{2}\right)\cos\left(\frac{2n - 1}{2} \pi x\right)\cos\left(\frac{2n - 1}{2} \pi \xi\right)}{\left(2n - 1\right)^2\pi^2 + 1},
    % \end{align*}
    % como desejado.
\end{proof}

\end{document}
