\section*{Exercício 6}
\begin{definition}{Matrizes de Pauli}{pauli}
    As \emph{matrizes de Pauli} são definidas por
    \begin{equation*}
        \sigma_1 = \begin{pmatrix}
            0&1\\1&0
        \end{pmatrix},\quad\sigma_2 = \begin{pmatrix}
            0&-i\\i&0
        \end{pmatrix},\quad\text{e}\quad\sigma_3 = \begin{pmatrix}
            1&0\\0&-1
        \end{pmatrix}.
    \end{equation*}
\end{definition}
\begin{proposition}{Relações algébricas das matrizes de Pauli}{pauli}
    As matrizes de Pauli satisfazem as relações
    \begin{equation*}
        [\sigma_a,\sigma_b] = 2i \sum_{c = 1}^3 \epsilon_{abc}\sigma_c,\quad
        \{\sigma_a, \sigma_b\} = 2\delta_{ab} \dsone,\quad\text{e}\quad
        \sigma_a \sigma_b = \delta_{ab} \dsone + i \sum_{c = 1}^3 \epsilon_{abc}\sigma_c,
    \end{equation*}
    onde \(\epsilon_{abc}\) é o símbolo de Levi-Civita e \(\dsone\) é a matriz identidade.
\end{proposition}
\begin{proof}
    Calculando os nove possíveis produtos, temos
    \begin{equation*}
        \sigma_1 \sigma_1 = \sigma_2 \sigma_2 = \sigma_3 \sigma_3 = \begin{pmatrix}
            1&0\\0&1
        \end{pmatrix} = \dsone,
    \end{equation*}
    \begin{align*}
        \sigma_1 \sigma_2 &= \begin{pmatrix}
            i&0\\0&-i
        \end{pmatrix} = i \sigma_3,&
        \sigma_2 \sigma_1 &= \begin{pmatrix}
            -i&0\\0&i
        \end{pmatrix} = -i \sigma_3,&
            [\sigma_1, \sigma_2] &= 2i \sigma_3,\\
        \sigma_1 \sigma_3 &= \begin{pmatrix}
            0&-1\\1&0
        \end{pmatrix} = -i \sigma_2,&
        \sigma_3 \sigma_1 &= \begin{pmatrix}
            0&1\\-1&0
        \end{pmatrix} = i \sigma_2,&
            [\sigma_1, \sigma_3] &= -2i \sigma_2,\\
        \sigma_2 \sigma_3 &= \begin{pmatrix}
            0&i\\i&0
        \end{pmatrix} = i \sigma_1,&
        \sigma_2 \sigma_1 &= \begin{pmatrix}
            0&-i\\-i&0
        \end{pmatrix} = -i \sigma_1,&
            [\sigma_2, \sigma_3] &= 2i \sigma_1,\\
    \end{align*}
    portanto segue que
    \begin{equation*}
        [\sigma_a, \sigma_b] = 2i \sum_{c=1}^3 \epsilon_{abc}\sigma_c\quad\text{e}\quad\{\sigma_a, \sigma_b\} = 2 \delta_{ab} \dsone.
    \end{equation*}
    Somando estas duas expressões, obtemos
    \begin{equation*}
        2\sigma_a \sigma_b = 2 \delta_{ab} \dsone + 2i\sum_{c = 1}^3 \epsilon_{abc}\sigma_c,
    \end{equation*}
    como desejado.
\end{proof}

\begin{proposition}{Base de \(M_2(\mathbb{C})\)}{base}
    O conjunto \(\set{\dsone, \sigma_1, \sigma_2, \sigma_3}\) é uma base para o espaço vetorial \(M_2(\mathbb{C})\) de matrizes quadradas \(2\times 2\) de coeficientes complexos.
\end{proposition}
\begin{proof}
    Mostremos que este conjunto é linearmente independente. Consideremos a combinação linear nula,
    \begin{equation*}
        \alpha_0 \dsone + \sum_{k=1}^3\alpha_k \sigma_k = 0.
    \end{equation*}
    Os coeficientes da combinação linear são encontrados pelo sistema de equações
    \begin{equation*}
        \begin{cases}
            \alpha_0 + \alpha_3 = 0\\
            \alpha_1 - i \alpha_2 = 0\\
            \alpha_1 + i \alpha_2 = 0\\
            \alpha_0 - \alpha_3 = 0
        \end{cases}
    \end{equation*}
    cuja única solução é \(\alpha_0 = \alpha_1 = \alpha_2 = \alpha_3 = 0\).

    Mostremos que este conjunto gera \(M_2(\mathbb{C})\). Para uma matriz arbitrária, temos
    \begin{equation*}
        \beta_0 \dsone + \sum_{k = 0}^3 \beta_k \sigma_k = \begin{pmatrix}
            z_{11}&z_{12}\\z_{21}&z_{22}
        \end{pmatrix}
        \iff
        \begin{cases}
            \beta_0 + \beta_3 = z_{11}\\
            \beta_1 - i \beta_2 = z_{12}\\
            \beta_1 + i \beta_2 = z_{21}\\
            \beta_0 - \beta_3 = z_{22}
        \end{cases}
    \end{equation*}
    cuja solução única,
    \begin{align*}
        \beta_0 &= \frac{z_{11} + z_{22}}{2},&
        \beta_1 &= \frac{z_{12} + z_{21}}{2},&
        \beta_2 &= \frac{z_{21} - z_{12}}{2i},&
        \beta_3 &= \frac{z_{11} - z_{22}}{2},
    \end{align*}
    é sempre definida para todos \(z_{11}, z_{12}, z_{21}, z_{22} \in \mathbb{C}\).
\end{proof}

\begin{proposition}{Exponencial de um vetor de Pauli}{exponencial_pauli}
    Seja \(\vec\eta = \eta_1 \vetor{e}_x + \eta_2\vetor{e}_y + \eta_3 \vetor{e}_z\) um vetor unitário de \(\mathbb{R}^3\). Seja o \emph{vetor de Pauli} a combinação linear
    \begin{equation*}
        \vec\eta \cdot \vec\sigma = \eta_1 \sigma_1 + \eta_2 \sigma_2 + \eta_3 \sigma_3,
    \end{equation*}
    então
    \begin{equation*}
        \exp(i \theta \vec\eta \cdot \vec\sigma) = (\cos\theta)\dsone + (i\sin\theta)(\vec\eta \cdot\vec\sigma),
    \end{equation*}
    para todo \(\theta\in \mathbb{C}.\)
\end{proposition}
\begin{proof}
    Pela \cref{prop:pauli}, temos
    \begin{align*}
        (\vec \eta \cdot \vec \sigma)^2 &= \sum_{k=1}^3 \eta_k \sigma_k \sum_{\ell = 1}^3 \eta_\ell \sigma_\ell\\
                                        &= \sum_{k=1}^3\sum_{\ell=1}^3 \eta_k \eta_\ell \left(\delta_{k\ell}\dsone + i \sum_{m = 1}^3 \epsilon_{k\ell m}\sigma_m\right)\\
                                        &= \sum_{k=1}^3 (\eta_k)^2 \dsone + i\sum_{k=1}^3\sum_{\ell=1}^3 \sum_{m = 1}^3 \eta_k \eta_\ell\epsilon_{k\ell m}\sigma_m\\
                                        &= \dsone + \frac12 \sum_{k=1}^3\sum_{\ell=1}^3 \eta_k \eta_\ell [\sigma_k, \sigma_\ell]\\
                                        &= \dsone + \frac12 \sum_{k=1}^3 [\eta_k \sigma_k, \vec \eta \cdot \vec \sigma]\\
                                        &= \dsone + \frac12 [\vec \eta\cdot \vec \sigma, \vec \eta\cdot \vec \sigma]\\
                                        &= \dsone,
    \end{align*}
    onde usamos a bilinearidade e anticomutatividade do comutador.

    Mostremos por indução em \(m \in \mathbb{N}\) que
    \begin{equation*}
        (i\theta\vec \eta \cdot \vec \sigma)^{2m} = (-1)^m\theta^{2m}\dsone\quad\text{e}\quad(i\theta\vec \eta \cdot \vec \sigma)^{2m+1} = (-1)^{m}i\theta^{2m+1}\vec \eta \cdot \vec \sigma,
    \end{equation*}
    para todo \(\theta \in \mathbb{C}\). Essas igualdades seguem trivialmente para \(m = 0\) e para \(m = 1\) temos
    \begin{equation*}
        (i\theta \vec \eta\cdot \vec \sigma)^2 = - \theta^2 \dsone\quad\text{e}\quad(i\theta\vec\eta\cdot\vec\sigma)^3 = -i\theta^3 \vec \eta \cdot \vec \sigma,
    \end{equation*}
    como proposto. Suponhamos que as igualdades sejam satisfeitas para algum \(k \in \mathbb{N}\), então
    \begin{align*}
        (i\theta\vec \eta \cdot \vec \sigma)^{2k+2} &= (i\theta\vec \eta \cdot \vec \sigma)(i\theta\vec \eta \cdot \vec \sigma)^{2k+1}\\
                                                    &= (i\theta\vec \eta \cdot \vec \sigma)\left[(-1)^ki\theta^{2k+1}\vec \eta\cdot \vec \sigma\right]\\
                                                    &= (-1)^{k+1}\theta^{2k+2} \dsone
    \end{align*}
    e
    \begin{align*}
        (i\theta\vec \eta \cdot \vec \sigma)^{2k+3} &= (i\theta\vec \eta \cdot \vec \sigma)(i\theta\vec \eta \cdot \vec \sigma)^{2k+2}\\
                                                    &= (i\theta\vec \eta \cdot \vec \sigma)\left[(-1)^{k+1}\theta^{2k+2} \dsone\right]\\
                                                    &= (-1)^{k+1}i\theta^{2k+3}\vec \eta \cdot \vec \sigma,
    \end{align*}
    isto é, as igualdades são satisfeitas por \(k + 1\). Pelo princípio da indução finita, são válidas para todo \(m \in \mathbb{N}\).

    Assim, para todo \(\theta \in \mathbb{C}\), temos
    \begin{align*}
        \exp(i\theta \vec \eta \cdot \vec \sigma) &= \sum_{\ell = 0}^{\infty} \frac{(i \theta \vec \eta \cdot \vec \sigma)^n}{n!}\\
                                                  &= \left[\sum_{m = 0}^{\infty} \frac{(-1)^{m}\theta^{2m}}{(2m)!}\right] \dsone + \left[\sum_{m=0}^\infty\frac{(-1)^m \theta^{2m+1}}{(2m+1)!}\right] i \vec \eta \cdot \vec \sigma\\
                                                  &= (\cos\theta)\dsone + (i\sin\theta)(\vec \eta \cdot \vec \sigma),
    \end{align*}
    como desejado.
\end{proof}

\begin{proposition}{Solução de uma classe de equações diferenciais}{ex6}
    Seja o operador constante
    \begin{equation*}
        H = B_1 \sigma_1 + B_2 \sigma_2 + B_3 \sigma_3,
    \end{equation*}
    com \(B_1, B_2, B_3 \in \mathbb{R}\). A solução da equação diferencial
    \begin{equation*}
        i \diff{\psi}{t} = H\psi(t),
    \end{equation*}
    onde \(\psi(t) = \begin{smallpmatrix}
        \psi_1(t)\\\psi_2(t)
    \end{smallpmatrix}\in \mathbb{C}^2\) satisfazendo a condição inicial \(\psi(0) = \psi_0 = \begin{smallpmatrix}
       \psi_1^0 \\\psi_2^0
    \end{smallpmatrix}\), é
    \begin{equation*}
        \psi(t) = \begin{cases}
            \psi_0,&\text{se }H=0\\
            \cos(Bt) \psi_0 -\frac{i\sin(Bt)}{B} H\psi_0,&\text{se }H\neq0
        \end{cases}
    \end{equation*}
    para todo \(t \in \mathbb{R}\), onde \(B = \sqrt{B_1^2 + B_2^2 + B_3^2}\).
\end{proposition}
\begin{proof}
    A equação diferencial pode ser escrita como
    \begin{equation*}
        \diff{\psi}{t} = -i H\psi(t),
    \end{equation*}
    portanto a solução do problema de valor inicial é dada por
    \begin{equation*}
        \psi(t) = \exp(-it H)\psi_0,
    \end{equation*}
    para todo \(t \in \mathbb{R}\). Para \(H = 0\), temos a solução constante \(\psi(t) = \psi_0\) para todo \(t \in \mathbb{R}\).

    Suponhamos que \(H\) não seja o operador nulo. Seja \(B = \sqrt{B_1^2 + B_2^2 + B_3^2}\) e \(\vec \eta \in \mathbb{R}^3\) tal que
    \begin{equation*}
        B \vec \eta = B_1 \vetor{e}_x + B_2\vetor{e}_y + B_3 \vetor{e}_z.
    \end{equation*}
    Assim, temos \(H = B \vec \eta \cdot \vec \sigma\). Com isso, pela \cref{prop:exponencial_pauli}, segue que
    \begin{equation*}
        \exp(-itH) = \cos(Bt) \dsone - (i \sin(Bt)) (\vec \eta \cdot \vec \sigma) = \cos(Bt) \dsone - \frac{i\sin(Bt)}{B} H.
    \end{equation*}
    Logo, a solução da equação diferencial é
    \begin{equation*}
        \psi(t) = \cos(Bt) \psi_0 - \frac{i\sin(Bt)}{B}H\psi_0,
    \end{equation*}
    para todo \(t \in \mathbb{R}\).
\end{proof}
