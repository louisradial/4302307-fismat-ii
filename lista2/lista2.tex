\documentclass[12pt,a4paper]{article}

% Language and formatting
\usepackage{polyglossia}
\usepackage{csquotes} %fvextra to avoid warning?
\setmainlanguage[variant=brazilian]{portuguese}

% \usepackage[backend=biber, style=alphabetic, sorting=ynt]{biblatex}
% \addbibresource{bibliography.bib}

% \setmainfont{Palatino Linotype}
% \setmathfont{Palatino Linotype}
\usepackage[a4paper, margin=1.5cm]{geometry}
\usepackage{booktabs}

% % title header
% \usepackage{titleps}% http://ctan.org/pkg/titleps
% \makeatletter
% \newpagestyle{main}{% Define page style main
%     \sethead%
%     [\textbf\thepage][][\thechapter.\ \chaptertitle]% [<even-left>][<even-center>][<even-right>]
%     {\thesection.\ \sectiontitle}{}{\textbf\thepage}% {<odd-left>}{<odd-center>}{<odd-right>}
%     \setfoot{}{}{}% {<left>}{<center>}{<right>}
% }
% \pagestyle{main}% Use page style main

% Images
\usepackage{tikz,pgfplots}
\pgfplotsset{compat=1.18}
\usetikzlibrary{cd}
\usepackage{graphicx, caption, subcaption}
\usepackage{float}

% Math tools
\usepackage{amsfonts, mathtools, amssymb, amsmath, amsthm, enumitem, dsfont}
\usepackage{newpxtext, newpxmath}
% \numberwithin{equation}{section}
\usepackage[ISO]{diffcoeff}
\usepackage{tensor}
\usepackage{siunitx}

% Misc
\usepackage{luacolor}
\usepackage[breakable]{tcolorbox}

\difdef{fp}{}{
    outer-Ldelim = \left.,
    outer-Rdelim = \right|,
    sub-nudge=0 mu
}
\newcommand\todo[1][!]{{\color{Red} TODO: {#1}}}
\difdef{l}{i}{outer-Rdelim = \,, outer-Ldelim=}
\difdef{l}{dn}{style=d^}
\NewDocumentCommand\dli{}{\dl.i.}
\DeclareMathOperator\Riem{Riem}
\DeclareMathOperator\Orb{Orb}
\DeclareMathOperator\Stab{Stab}
\DeclareMathOperator\sgn{sgn}
\DeclareMathOperator\End{End}
\DeclareMathOperator\tr{tr}
\DeclareMathOperator\sech{sech}
\DeclareMathOperator\artanh{artanh}
\DeclareMathOperator\hor{hor}
\DeclareMathOperator\ver{ver}
\DeclarePairedDelimiter\abs{\lvert}{\rvert}
\DeclarePairedDelimiter\norm{\lVert}{\rVert}
\DeclarePairedDelimiterX\inner[2]{\langle}{\rangle}{#1,\mathopen{}#2}
\DeclarePairedDelimiter\set{\{}{\}}

\newcommand\dsone{\ensuremath{\mathds1}}
\newenvironment{smallpmatrix}{\left(\begin{smallmatrix}}{\end{smallmatrix}\right)}
\newcommand\ract{\mathbin{\vartriangleleft}}
\newcommand\ractalt{\mathbin{\blacktriangleleft}}
\newcommand\lact{\mathbin{\vartriangleright}}
\newcommand\lactalt{\mathbin{\blacktriangleright}}
\newcommand\ad[1]{\operatorname{ad}_{#1}}
\newcommand\Ad[1]{\operatorname{Ad}_{#1}}
\newcommand\preim[2]{\operatorname{preim}_{#1}{\left(#2\right)}}
\newcommand\id[1]{\operatorname{id}_{#1}}
\newcommand\colorunderline[2]{{\color{#1}\underline{{\color{black}{#2}}}}}
\newcommand\Hom[2][]{\ensuremath{\operatorname{Hom}_{#1}{\left(#2\right)}}}
\newcommand\bundle[3]{\ensuremath{#1 \mathrel{\overset{#2}{\to}} #3}}
\newcommand\smooth[1]{\ensuremath{\mathcal{C}^\infty(#1)}}
\newcommand\sections[1]{\ensuremath{\Gamma\left(#1\right)}}
\newcommand\forms[2][]{\ensuremath{\Lambda^{#1}{\left({#2}\right)}}}
\newcommand\ffamily[3]{\ensuremath{\set*{#1}_{#2}^{#3}}}
\newcommand\family[2]{\ensuremath{\set*{#1}_{#2}}}
\newcommand\vetor[1]{\ensuremath{\boldsymbol{#1}}}
\newcommand\linear{\ensuremath{\mathrel{\tilde{\to}}}}
\newcommand\topology[1]{\ensuremath{\left(#1, \mathcal{O}_{#1}\right)}}
\newcommand\manifold[1]{\ensuremath{\left(#1, \mathcal{O}_{#1}, \mathscr{A}_{#1}\right)}}
\newcommand\restrict[2]{\ensuremath{\left.#1\right\rvert_{#2}}}
\newcommand\bfield[1]{\ensuremath{\diffp*{}{#1}}}
\newcommand\bvec[3][]{\ensuremath{\diffp*{#1}{#2}[#3]}}
\newcommand\bset[3]{\ensuremath{\set*{\diffp*{}{{#1}^1}[#3], \dots, \diffp*{}{{#1}^{#2}}[#3]}}}
\newcommand\pf[2][]{\ensuremath{{#2}_{\ast{#1}}}}
\newcommand\pb[2][]{\ensuremath{{#2}^{\ast}_{#1}}}

% catppuccin (latte)
\definecolor{Rosewater}{RGB}{220,138,120}
\definecolor{Flamingo}{RGB}{221,120,120}
\definecolor{Pink}{RGB}{234,118,203}
\definecolor{Mauve}{RGB}{136,57,239}
\definecolor{Red}{RGB}{210,15,57}
\definecolor{Maroon}{RGB}{230,69,83}
\definecolor{Peach}{RGB}{254,100,11}
\definecolor{Yellow}{RGB}{223,142,29}
\definecolor{Green}{RGB}{64,160,43}
\definecolor{Teal}{RGB}{23,146,153}
\definecolor{Sky}{RGB}{4,165,229}
\definecolor{Sapphire}{RGB}{32,159,181}
\definecolor{Blue}{RGB}{30,102,245}
\definecolor{Lavender}{RGB}{114,135,253}
\definecolor{Text}{RGB}{76,79,105}
\definecolor{Subtext1}{RGB}{92,95,119}
\definecolor{Subtext0}{RGB}{108,111,133}
\definecolor{Overlay2}{RGB}{124,127,147}
\definecolor{Overlay1}{RGB}{140,143,161}
\definecolor{Overlay0}{RGB}{156,160,176}
\definecolor{Surface2}{RGB}{172,176,190}
\definecolor{Surface1}{RGB}{188,192,204}
\definecolor{Surface0}{RGB}{204,208,218}
\definecolor{Base}{RGB}{239,241,245}
\definecolor{Mantle}{RGB}{230,233,239}
\definecolor{Crust}{RGB}{220,224,232}

% References
\usepackage{hyperref}
\usepackage[brazilian,capitalize,nameinlink,noabbrev]{cleveref}
% \captionsetup[figure]{hypcap=false}
\makeatletter
\hypersetup{
    pdftitle=\@title,
    pdfauthor=\@author,
    colorlinks=true,
    linkcolor=Mauve,
    citecolor=pink,
    filecolor=red,
    urlcolor=blue,
    bookmarksdepth=4
}
\makeatother

% tcolorbox environments
\tcbuselibrary{theorems}
% theorem
\newtcbtheorem[auto counter, crefname={Teorema}{Teoremas}]{theorem}{Teorema}%
{breakable,colback=Mauve!5,colframe=Mauve!95!black,fonttitle=\bfseries}{thm}

% definition
\newtcbtheorem[auto counter, crefname={Definição}{Definições}]{definition}{Definição}%
{breakable, colback=Pink!5,colframe=Pink!95!black,fonttitle=\bfseries}{def}

% proposition
\newtcbtheorem[auto counter, crefname={Proposição}{Proposições}]{proposition}{Proposição}%
{breakable,colback=Lavender!5,colframe=Lavender!95!black,fonttitle=\bfseries}{prop}

% lemma
\newtcbtheorem[auto counter, crefname={Lema}{Lemas}]{lemma}{Lema}%
{breakable,colback=Flamingo!5,colframe=Flamingo!95!black,fonttitle=\bfseries}{lem}

% exercício
\newtcbtheorem[auto counter, crefname={Exercício}{Exercícios}]{exercício}{Exercício}%
{breakable,colback=Sky!5,colframe=Sky!95!black,fonttitle=\bfseries}{ex}

\title{Física Matemática II\\Segunda Lista de Exercícios}
\author{Louis Bergamo Radial\\8992822}

\begin{document}
\maketitle
\section*{Exercício 1}
\begin{proposition}{Métrica trivial}{métrica_trivial}
    Seja \(X\) um conjunto não vazio, então \((X, d_\mathrm{t})\) é um espaço métrico, onde a função \(d_\mathrm{t} : X \times X \to \mathbb{R}\) é a métrica trivial, definida por
    \begin{equation*}
        d_\mathrm{t}(x,y) = \begin{cases}
            0, & \text{se }x=y,\\
            1, & \text{se }x\neq y,\\
        \end{cases}
    \end{equation*}
    para todo \(x,y \in X\).
\end{proposition}
\begin{proof}
    Pela definição da métrica trivial, temos
    \begin{equation*}
        d_\mathrm{t}(x,y) = 0 \iff x = y
    \end{equation*}
    para todo \(x,y \in X\). De mesma forma, pela simetria de relação de igualdade, temos
    \begin{equation*}
        d_\mathrm{t}(x,y) = d_\mathrm{t}(y,x).
    \end{equation*}
    Ainda, a imagem da função \(d_\mathrm{t}\) é contida na semirreta \([0,\infty)\),
    \begin{equation*}
        d_\mathrm{t}(X\times X) = \set*{0, 1} \subset [0, \infty).
    \end{equation*}
    Assim, resta mostrar que a métrica trivial satisfaz a desigualdade triangular.

    Consideremos \(x,y,z \in \mathbb{R}\), então segue que
    \begin{equation*}
        0 \leq d_\mathrm{t}(x,z) + d_\mathrm{t}(z,y) \leq 2,
    \end{equation*}
    com os únicos valores possíveis para a soma sendo \(\set{0,1,2}\). No caso em que \(x = y\), temos \(d_\mathrm{t}(x,y) = 0\), portanto
    \begin{equation*}
        d_\mathrm{t}(x,y) \leq d_\mathrm{t}(x,z) + d_\mathrm{t}(z,y)
    \end{equation*}
    é satisfeita de forma trivial. No caso em que \(x \neq y\), temos \(d_\mathrm{t}(x,y) = 1\), portanto pela transitividade da igualdade temos que
    \begin{equation*}
        1 \leq d_\mathrm{t}(x,z) + d_\mathrm{t}(z,y) \leq 2,
    \end{equation*}
    já que \(z\) não pode ser igual a tanto \(x\) quanto \(y\), de modo que
    \begin{equation*}
        d_\mathrm{t}(x,y) \leq d_\mathrm{t}(x,z) + d_\mathrm{t}(z,y).
    \end{equation*}
    Dessa forma, mostramos que a desigualdade triangular é satisfeita em todos os casos, portanto \((X,d_\mathrm{t})\) é um espaço métrico.
\end{proof}

\section*{Exercício 2}
\begin{corollary}\label{exercício2a}
    A função de Green do problema de Sturm \(u''(x) = f(x)\) onde \(u\) é definida no intervalo \(x \in [0,1]\) e satisfaz \(u'(0) = 0\) e \(u(1) = 0\) é dada por
    \begin{equation*}
        G(x,\xi) = \left\{\begin{aligned}
                \xi - 1, && 0 \leq x < \xi \leq 1\\
                x - 1, && 0 \leq \xi < x \leq 1
        \end{aligned}\right.
    \end{equation*}
\end{corollary}
\begin{proof}
    Identificando \(\alpha_1 = 0, \alpha_2 = 1, \beta_1 = 1,\) e \(\beta_2 = 0,\) o resultado segue da \cref{prop:exercício1}.
\end{proof}

\begin{proposition}{Autovalores e autofunções do problema de Sturm-Liouville \(u'' + \lambda u = 0\)}{exercício2b}
    Os autovalores e as autofunções normalizadas do problema de Sturm-Liouville
    \begin{equation*}
        u'' + \lambda u = 0,
    \end{equation*}
    onde \(u\) é definida no intervalo \(x \in [0,1]\) e satisfaz as condições de contorno \(u'(0) = 0\) e \(u(1) = 0\) são dados por
    \begin{equation*}
        \lambda_n = \left(n-\frac{1}{2}\right)^2\pi^2\quad\text{e}\quad
        u_n =\sqrt{2} \cos\left(\frac{2n-1}{2}\pi x\right),
    \end{equation*}
    para todo \(n \in \mathbb{N}\smallsetminus\set{0}.\)
\end{proposition}
\begin{proof}
    Notamos que a solução geral da equação diferencial do problema de Sturm-Liouville é
    \begin{equation*}
        u(x) = \begin{cases}
            A\cosh(\sqrt{-\lambda}x) + B\sinh(\sqrt{-\lambda}x), & \lambda < 0\\
            Ax + B, &\lambda = 0\\
            A\cos(\sqrt{\lambda}x)+ B\sin(\sqrt{\lambda}x),&\lambda > 0\\
        \end{cases}.
    \end{equation*}
    Notemos que pelas condições de contorno, a solução para \(\lambda = 0\) é a solução trivial, portanto podemos descartar este caso. Para \(\lambda < 0\), segue de \(u'(0) = 0\) que \(B = 0\), então como o cosseno hiperbólico tem imagem positiva, a única solução de \(u(1) = 0\) é \(A = 0\), isto é, este caso também leva apenas a soluções triviais. Nos resta apenas o caso \(\lambda > 0\), temos de \(u'(0) = 0\) que \(B = 0\), logo da outra condição de contorno obtemos
    \begin{equation*}
        u(1) = 0 \implies \sqrt{\lambda} = \left(n-\frac12\right)\pi,\quad\text{com}\quad n \in \mathbb{N}\smallsetminus\set{0}.
    \end{equation*}
    Deste modo, os autovalores do problema de Sturm-Liouville considerado são
    \begin{equation*}
        \lambda_n = \left(n-\frac12\right)^2\pi^2,
    \end{equation*}
    para \(n \in \mathbb{N}\smallsetminus\set{0}\).

    Para determinar as autofunções normalizadas, notamos que o produto interno para este problema de Sturm-Liouville coincide com o produto interno usual para o espaço de funções integráveis em \([0,1]\). Impondo que \(\inner{u_n}{u_n} = 1\), obtemos
    \begin{equation*}
        \int_{0}^{1} \dli{x}\abs{A}^2 \cos^2\left(\frac{2n-1}{2}\pi x\right) = 1 \implies \abs{A}^2 \int_{0}^{1}\dli{x} \frac{1 + \cos((2n-1)\pi x)}{2} = 1 \implies \abs{A} = \sqrt{2},
    \end{equation*}
    portanto as autofunções normalizadas do problema de Sturm-Liouville são
    \begin{equation*}
        u_n(x) = \sqrt{2} \cos\left(\frac{2n-1}{2}\pi x\right),
    \end{equation*}
    para \(n \in \mathbb{N}\smallsetminus\set{0}.\)
\end{proof}

\begin{corollary}\label{exercício2c}
    A função de Green para o problema de Sturm associado é dada por
    \begin{equation*}
        G(x, \xi) = - \frac{8}{\pi^2}\sum_{m = 0}^{\infty} \frac{\cos\left(\frac{2m + 1}{2}\pi x\right)\cos\left(\frac{2m + 1}{2}\pi \xi\right)}{(2m + 1)^2},
    \end{equation*}
    para todo \((x, \xi) \in [0,1]\times[0,1]\).
\end{corollary}
\begin{proof}
    Pela fórmula de Mercer, temos
    \begin{equation*}
        G(x,\xi) = - \sum_{n = 1}^{\infty} \frac{u_n(x) u_n(\xi)}{\lambda_n},
    \end{equation*}
    onde \(u_n\) é a autofunção normalizada associada ao autovalor \(\lambda_n\) do problema de Sturm-Liouville. Pela \cref{prop:exercício2b}, segue que
    \begin{align*}
        G(x, \xi) &= - \sum_{n = 1}^\infty \frac{\left[\sqrt{2}\cos\left(\frac{2n-1}{2}\pi x\right)\right]\left[\sqrt{2}\cos\left(\frac{2n-1}{2}\pi \xi\right)\right]}{\left(\frac{2n-1}{2}\right)^2\pi^2}\\
                  &= - \frac{8}{\pi^2} \sum_{n = 1}^\infty \frac{\cos\left(\frac{2n-1}{2}\pi x\right)\cos\left(\frac{2n-1}{2}\pi \xi\right)}{(2n-1)^2}.
    \end{align*}
    Fazendo a troca de variável de soma \(m = n - 1\), obtemos a expressão desejada.
\end{proof}
\begin{corollary}
    Vale a identidade
    \begin{equation*}
        \frac{\pi^2}{8} = \sum_{m = 0}^\infty \frac{1}{(2m+1)^2}.
    \end{equation*}
\end{corollary}
\begin{proof}
    Pelo \cref{exercício2a} e pela continuidade da função de Green, segue que \(G(0,0) = -1\). Desse modo, pelo \cref{exercício2c}, temos
    \begin{equation*}
        -\frac{8}{\pi^2} \sum_{m=0}^\infty \frac{1}{(2m+1)^2} = -1\implies
        \sum_{m=0}^\infty \frac{1}{(2m+1)^2} = \frac{\pi^2}{8},
    \end{equation*}
    como desejado.
\end{proof}

\begin{proposition}{Solução para o problema de Sturm \(u''(x) = (3 - x)e^x\)}{exercício2d}
    A solução do problema de Sturm
    \begin{equation*}
        u''(x) = (3-x)e^x
    \end{equation*}
    com condições de contorno \(u'(0) = 0\) e \(u(1) = 0\) é
    \begin{equation*}
        u(x) =
    \end{equation*}
\end{proposition}

\section*{Exercício 3}
\begin{definition}{Mapa logístico}{mapa_logístico}
    A aplicação
    \begin{align*}
        T_a : \mathbb{R} &\to \mathbb{R}\\
                       x &\mapsto ax(1 - x)
    \end{align*}
    é chamada de \emph{mapa logístico} ao parâmetro \(a \in \mathbb{R}\).
\end{definition}

\begin{proposition}{Pontos fixos do mapa logístico}{pontos_fixo_mapa_logístico}
    Os pontos fixos do mapa logístico \(T_a\) são dados por
    \begin{equation*}
        x^\alpha = 0 \quad\text{e}\quad x^\beta = \frac{a-1}{a},
    \end{equation*}
    onde \(x^\beta\) claramente só está definido para \(a \neq 0\). O ponto fixo \(x^\beta\) pertence a \([0,1]\) se e somente se \(a \geq 1\).
\end{proposition}
\begin{proof}
    A equação de ponto fixo para \(T_a\) é dada por
    \begin{equation*}
        x = ax(1 - x) \implies x(a - 1 - ax) = 0,
    \end{equation*}
    cujas soluções são justamente \(x^\alpha\) e \(x^\beta\), com \(x^\beta\) definido apenas para \(a \neq 0\).

    Notemos que \(x^\beta = 1 - \frac1a\), portanto para \(a \geq 1\), temos \(x^\beta \in [0,1) \subset [0,1]\), uma vez que \(x^\beta\) é crescente para \(a > 0\). Para \(x^\beta \in [0,1]\), temos
    \begin{align*}
        x^\beta \in [0,1] &\implies 1 - \frac{1}{a} \geq 0 \land 1 - \frac{1}{a} \leq 1\\
                          &\implies a \notin [0, 1) \land a \geq 1\\
                          &\implies a \geq 1,
    \end{align*}
    como desejado.
\end{proof}

\begin{proposition}{Restrição do mapa logístico}{restrição_logístico}
    Seja \(A = [0,1].\) Se \(a \in [0,4]\), a aplicação \(\restrict{T_a}{A} : A \to \mathbb{R}\) é um endomorfismo.
\end{proposition}
\begin{proof}
    Trivialmente, se \(a = 0\) então \(T_a(\mathbb{R}) = \set{0} \subset A\), logo \(\restrict{T_0}{A} : A \to A\). Assim, podemos supor \(a \neq 0\).

    Como \(T_a\) é uma função suave, pelo teorema de Weierstrass esta função admite valor máximo e mínimo no compacto \(A\). Como
    \begin{equation*}
        \diff{T_a}{x} = 0 \implies x = \frac12 \in A,
    \end{equation*}
    segue que os valores de máximo e mínimo de \(T_a\) em \(A\) só podem ocorrer em \(x = 0, x = 1\) e \(x = \frac12\), cujos valores são \(T_a(0) = T_a(1) = 0\) e \(T_a(\frac12) = \frac{a}4\). Desse modo, para \(a > 0\) temos que o máximo global de \(\restrict{T_a}{A}\) ocorre em \(x = \frac12\). Assim, segue que
    \begin{equation*}
        a \in (0, 4] \implies 0 \leq T_a(x) \leq \frac{a}{4} \leq 1
    \end{equation*}
    para todo \(x \in A\). Concluímos portanto que \(T_a(A) \subset A\) para \(a \in [0,4]\).
\end{proof}

\begin{proposition}{Pontos fixos da restrição do mapa logístico}{pontos_fixos_restrição}
    Para \(a \in [0,1],\) a aplicação \(\restrict{T_a}{A} : A \to A\) tem um único ponto fixo, a saber, \(x = 0\). Para \(a \in (1, 4],\) a aplicação apresenta dois pontos fixos distintos, \(x = 0\) e \(x = x^\beta\).
\end{proposition}
\begin{proof}
    Para \(a = 0\), a imagem da aplicação é o conjunto \(\set{0}\), portanto o único ponto fixo é \(x = 0\).

    Consideremos \(a \in (0, 4]\). Pela \cref{prop:pontos_fixo_mapa_logístico}, os pontos fixos de \(T_a : \mathbb{R} \to \mathbb{R}\) são \(x^\alpha = 0\) e \(x^\beta\), com \(x^\beta \in A\iff a \geq 1\). Desse modo, para \(a \in (0, 1),\) o único ponto fixo de \(\restrict{T_a}{A}: A \to A\) é \(x = 0\). Ainda, para \(a = 1, x^\beta = 0\), de modo que para \(a \in [0,1], \) temos o único ponto fixo \(x = 0\) em \(A\). Para \(a \in (1, 4],\) \(x^\beta \neq 0,\) de modo que \(\restrict{T_a}{A}\) apresente dois pontos fixos distintos em \(A\).
\end{proof}

\begin{proposition}{Condições para a restrição do mapa logístico ser uma contração}{contração_logístico}
    Para \(a \in [0,1),\) a aplicação \(\restrict{T_a}{A} : A \to A\) é uma contração. Para \(a \in (1, 4]\), a aplicação não é contrativa.
\end{proposition}
\begin{proof}

\end{proof}

\section*{Exercício 4}
\begin{proposition}{Autofunções do problema de Sturm-Liouville com \(u'' + u' + \lambda u = 0\)}{exercício4d}
    Os autovalores e as autofunções normalizadas do problema de Sturm-Liouville \(u'' + u' + \lambda u = 0\) com condições de contorno \(u(0) = 0\) e \(u'(1) = 0\) são
    \begin{equation*}
        \lambda_n =\left(n - \frac12\right)^2 \pi^2 + \frac14
        \quad\text{e}\quad
        u_n(x) =\sqrt{2} \exp\left(-\frac{1}{2}x\right)\cos\left(\frac{2n - 1}{2}\pi x\right)
    \end{equation*}
    para \(n \in \mathbb{N}\smallsetminus{0}\). Assim,
    \begin{equation*}
        G(x,\xi) = - 8 \sum_{n = 1}^\infty \frac{\exp\left(-\frac{x + \xi}{2}\right)\cos\left(\frac{2n - 1}{2} \pi x\right)\cos\left(\frac{2n - 1}{2} \pi \xi\right)}{\left(2n - 1\right)^2\pi^2 + 1},
    \end{equation*}
    para \((x,\xi) \in [0,1]\times[0,1]\), é a função de Green do problema de Sturm associado.
\end{proposition}
\begin{proof}
    Como já feito na \cref{prop:exercício3b}, temos as soluções gerais da equação diferencial, que dependem se \(\lambda\) é nulo ou se \(\lambda - \frac14\) é positivo, nulo, ou negativo. Para \(\lambda = 0\), temos \(u(x) = \alpha e^{-x} + \beta\), portanto de \(u'(1) = 0\), segue que \(\alpha = 0\), levando à solução trivial. Para \(\lambda = \frac14\), temos \(u(x) = (\alpha x + \beta)e^{-\frac12 x}\), logo de \(u(0) = 0\), segue que \(\beta = 0\), portanto
    \begin{equation*}
        u'(x) = \alpha \left(1 - \frac12 x\right)e^{-\frac12 x},
    \end{equation*}
    e temos de \(u'(1) = 0\) que \(\alpha = 0\). Para \(\lambda < \frac14\) e \(\lambda \neq 0\), temos a solução geral
    \begin{equation*}
        u(x) = \alpha \exp\left(\lambda_+ x\right) + \beta \exp\left(\lambda_-x\right),
    \end{equation*}
    onde \(\lambda_+ = \frac{-1 + \sqrt{1 - 4 \lambda}}{2}\) e \(\lambda_- = \frac{-1 - \sqrt{1 - 4 \lambda}}{2}\) são valores reais e distintos. Das condições de contorno, temos
    \begin{equation*}
        \begin{cases}
            \alpha + \beta = 0\\
            \lambda_+ \alpha e^{\lambda_+} + \lambda_- \beta e^{\lambda_-} = 0
        \end{cases} \implies
        \alpha (\lambda_+ e^{\lambda_+} - \lambda_- e^{\lambda_-}) = 0,
    \end{equation*}
    e \todo[mostrando que \(\lambda_+ e^{\lambda_+} - \lambda_- e^{\lambda_-}\) não se anula], segue que há apenas a solução trivial. Para \(\lambda > \frac14\), temos a solução geral
    \begin{equation*}
        u(x) = e^{-\frac12x}\left[\alpha \cos\left(x\sqrt{\lambda - \frac14}\right) + \beta \sin\left(x\sqrt{\lambda - \frac14}\right)\right],
    \end{equation*}
    portanto da condição de contorno \(u(0) = 0\), segue que \(\alpha = 0\), e da condição de contorno \(u'(1) = 0\), segue que as soluções não triviais devem satisfazer
    \begin{equation*}
        2 \sqrt{\lambda - \frac14}\cos\left(\sqrt{\lambda - \frac14}\right) - \sin\left(\sqrt{\lambda - \frac14}\right) = 0 \implies 2\sqrt{\lambda - \frac14} = \tan\left(\sqrt{\lambda - \frac14}\right).
    \end{equation*}
    Notemos que a equação transcendental obtida possui infinitas soluções para \(\lambda > \frac14\), uma vez que a imagem da tangente contém todos os números reais positivos e a imagem da raiz quadrada também. Seja \(\varphi_n\) a \(n\)-ésima solução positiva da equação transcendental \(2\varphi = \tan \varphi\), então os autovalores do problema de Sturm-Liouville são dados por
    \begin{equation*}
        \lambda_n = \varphi_n^2 + \frac14,
    \end{equation*}
    para \(n \in \mathbb{N} \smallsetminus\set{0}\).

    Notemos que o produto interno para este problema de Sturm-Liouville é dado por
    \begin{equation*}
        \inner{f}{g}_r = \int_{0}^{1} e^{t}\dli{t}\overline{f(t)}g(t)
    \end{equation*}
    para quaisquer funções integráveis \(f, g\) em \([0,1]\). Para determinar as autofunções, devemos impor que as soluções encontradas têm norma unitária em relação a este produto interno, isto é,
    \begin{align*}
        \inner{u_n}{u_n}_r = 1 &\implies \int_0^1 e^t \dli{t} \abs{\beta_n}^2 e^{-t}\cos^2\left(\varphi_n t\right) = 1\\
                               &\implies  \frac12\abs{\beta_n}^2 \left[\int_0^1 \dli{t} - \int_0^1 \dli{t} \cos\left(2\varphi_n t\right)\right] = 1
    \end{align*}
    % Deste modo,
    % \begin{equation*}
    %     u_n(x) = \sqrt{2} \exp\left(-\frac{1}{2}x\right)\cos\left(\frac{2n - 1}{2}\pi x\right)
    % \end{equation*}
    % são as autofunções para este problema de Sturm-Liouville, para \(n \in \mathbb{N}\smallsetminus\set{0}\).
    %
    % Pela fórmula de Mercer, a função de Green do problema associado é
    % \begin{align*}
    %     G(x,\xi) &= -\sum_{n = 1}^\infty \frac{u_n(x)u_n(\xi)}{\lambda_n}\\
    %              &= - 8 \sum_{n = 1}^\infty \frac{\exp\left(-\frac{x + \xi}{2}\right)\cos\left(\frac{2n - 1}{2} \pi x\right)\cos\left(\frac{2n - 1}{2} \pi \xi\right)}{\left(2n - 1\right)^2\pi^2 + 1},
    % \end{align*}
    % como desejado.
\end{proof}

\section*{Exercício 5}
\begin{proposition}{Operador de derivação no espaço de polinômios complexos}{derivaçãoPn}
    Seja \(\mathcal{P}_n\) o espaço vetorial complexo \((n + 1)\)-dimensional de todos os polinômios complexos de grau menor ou igual a \(n\). O operador diferencial nilpotente \(D = \diff*{}{x}\) pode ser representado matricialmente por
    \begin{equation*}
        D \doteq \begin{bmatrix}
            0 & 1 & 0 & \dots & 0\\
            0 & 0 & 1 & \dots & 0\\
            \vdots & \vdots & \vdots & \ddots & \vdots\\
            0 & 0 & 0 & \dots & 1\\
            0 & 0 & 0 & \dots & 0
        \end{bmatrix}
    \end{equation*}
    na base \(\set{e_0, e_1, \dots, e_n}\), onde \(e_k = \frac{x^k}{k!}\).
\end{proposition}
\begin{proof}
    Na base dada, temos \(D e_0 = \diff*{1}{x} = 0\) e para \(1 \leq k \leq n\)
    \begin{equation*}
        D e_k = \diff*{\frac{x^k}{k!}}{x} = \frac{x^{k-1}}{(k-1)!} = e_{k-1},
    \end{equation*}
    o que confirma a representação matricial afirmada.

    Pelo \cref{lem:derivada}, temos \(D^{n+1}e_k = 0\) para todo \(0 \leq k \leq n\). Dessa forma, para um polinômio \(p \in \mathcal{P}_n\) com \(p = \sum_{k = 0}^{n} p_k e_k\), temos
    \begin{equation*}
        D^{n+1} p = \sum_{k=0}^n p_k D^{n+1} e_k = 0.
    \end{equation*}
    Assim, segue que o operador \(D^{n+1}\) é o operador nulo, isto é, \(D\) é nilpotente.
\end{proof}
\begin{lemma}{Binômio de Newton}{binomio_newton}
    Para \(a,b \in \mathbb{C}\), segue que
    \begin{equation*}
        (a + b)^k = \sum_{\ell = 0}^{k} \binom{k}{\ell} a^{k - \ell}b^{\ell}
    \end{equation*}
    para todo \(k \in \mathbb{N}\).
\end{lemma}
\begin{proof}
    A identidade segue trivialmente para \(k = 0\) e para \(k = 1\) temos
    \begin{equation*}
        \binom{1}{0}a + \binom{1}{1}b = a + b,
    \end{equation*}
    portanto a igualdade é satisfeita. Suponhamos que a expressão seja válida para algum \(m\in \mathbb{N}\), então
    \begin{align*}
        (a + b)^{m+1} &= (a + b) (a+b)^m = (a + b)\sum_{\ell = 0}^{m}\binom{m}{\ell} a^{m-\ell}b^{\ell}\\
                      &= \sum_{\ell = 0}^m \binom{m}{\ell} a^{m+1-\ell}b^{\ell} + \sum_{\ell = 1}^{m+1} \binom{m}{\ell-1} a^{m+1-\ell}b^{\ell}\\
                      &= a^{m+1} + b^{m+1} + \sum_{\ell=1}^{m} \left[\binom{m}{\ell} + \binom{m}{\ell-1}\right]a^{m+1-\ell}b^{\ell}\\
                      &= \sum_{\ell=0}^m\binom{m+1}{\ell} a^{m+1-\ell}b^{\ell},
    \end{align*}
    pelo \cref{lem:pascal}, isto é, \(m+1\) também satisfaz a identidade. Dessa forma, pelo princípio da indução finita, segue que é válida para todo \(k \in \mathbb{N}\).
\end{proof}

\begin{proposition}{Exponencial do operador de derivação}{exponencial}
    Para \(t \in \mathbb{C}\), temos
    \begin{equation*}
        \exp{(tD)} \doteq \begin{bmatrix}
            1 & t & \frac{t^2}{2} & \dots & \frac{t^{n-1}}{(n-1)!} & \frac{t^n}{n!}\\
            0 & 1 & t & \dots & \frac{t^{n-2}}{(n-2)!} & \frac{t^{n-1}}{(n-1)!}\\
            0 & 0 & 1 & \dots & \frac{t^{n-3}}{(n-3)!} & \frac{t^{n-2}}{(n-2)!}\\
            \vdots & \vdots & \vdots & \ddots& \vdots & \vdots\\
            0 & 0 & 0 & \dots & 1 & t\\
            0 & 0 & 0 & \dots & 0 & 1
        \end{bmatrix}
    \end{equation*}
    como a representação matricial do operador \(\exp(tD)\) na base \(\set{e_0, e_1, \dots, e_n}\) de \(\mathcal{P}_n\). Este operador satisfaz
    \begin{equation*}
        \left[\exp{(tD)}p\right](x) = p(x+t),
    \end{equation*}
    para todo \(p \in \mathcal{P}_n\) e \(x \in \mathbb{C}\).
\end{proposition}
\begin{proof}
    Pelo resultado obtido no \cref{lem:derivada}, segue que
    \begin{equation*}
        D^{\ell} e_k = \begin{cases}
            e_{k - \ell}, & \text{se }k\geq\ell\\
            0,&\text{se }k<\ell
        \end{cases}
    \end{equation*}
    para todo \(0 \leq k \leq n\) e \(\ell \in \mathbb{N}\).

    Como \(D\) é nilpotente com \(D^{n+1} = 0\), segue que
    \begin{equation*}
        \exp{(tD)} = \id{\mathcal{P}_n} + \sum_{\ell = 1}^{n}\frac{t^\ell}{\ell!}D^\ell
    \end{equation*}
    para todo \(t \in \mathbb{C}\). Assim, para um vetor \(e_k\) da base, temos
    \begin{align*}
        \exp{(tD)}e_k &= e_k + \sum_{\ell = 1}^n \frac{t^\ell}{\ell!}D^\ell e_k = e_k + \sum_{\ell = 1}^k \frac{t^\ell}{\ell!}e_{k-\ell}\\
                      &= e_k + t e_{k-1} + \frac{t^2}{2} e_{k-2} + \dots + \frac{t^k}{k!}e_0,
    \end{align*}
    o que confirma a representação matricial apresentada.

    Notemos que para \(x \in \mathbb{C}\), temos
    \begin{align*}
        \left[\exp{(tD)}e_k\right](x) &= \sum_{\ell=0}^k\frac{t^\ell}{\ell!}e_{k-\ell}(x) = \sum_{\ell=0}^k\frac{t^{\ell}}{\ell!}\frac{x^{k-\ell}}{(k-\ell)!}\\
                                      &= \frac{1}{k!}\sum_{\ell=0}^k \binom{k}{l} t^\ell x^{k - \ell}\\
                                      &= \frac{(t+x)^k}{k!} = e_k(t+x),
    \end{align*}
    pelo \cref{lem:binomio_newton}, para todo \(0 \leq k \leq n\). Dessa forma, para um polinômio \(p \in \mathcal{P}_n\) qualquer, com \(p = \sum_{k = 0}^n p_k e_k\), temos
    \begin{equation*}
        \left[\exp{(tD)}p\right](x) = \sum_{k = 0}^n p_k \left[\exp{(tD)}e_k\right](x) = \sum_{k = 0}^n p_k e_k(t+x) = p(t + x),
    \end{equation*}
    como desejado.
\end{proof}

\section*{Exercício 6}
\begin{proposition}{\(\mathbb{Q}\) não é completo em relação à métrica usual}{cauchy_não_converge}
    A sequência \(\family{x_n}{n\in \mathbb{N}}\subset \mathbb{Q}\) definida por
    \begin{equation*}
        x_n = \sum_{k = 0}^n \frac{1}{k!}
    \end{equation*}
    é de Cauchy mas não converge a nenhum número racional em relação à métrica usual.
\end{proposition}
\begin{proof}
    Consideremos \(n,m \in \mathbb{N}\) com \(n > m\), então
    \begin{align*}
        \abs{x_n - x_m} &= \abs*{\sum_{k=0}^{n}\frac{1}{k!} - \sum_{j = 0}^m\frac{1}{j!}}\\
                        % &= \sum_{k=m+1}^{n} \frac{1}{k!}\\
                        &= \sum_{k=0}^{n-m-1} \frac{1}{(k+m+1)!}\\
                        &= \frac{1}{(m+1)!} \sum_{k=0}^{n-m-1} \frac{(m+1)!}{(k+m+1)!}\\
                        &= \frac{1}{(m+1)!} \left(1 + \frac{1}{m+2} + \frac{1}{(m+2)(m+3)} + \dots + \frac{(m+1)!}{n!}\right)\\
                        &\leq \frac{1}{(m+1)!}\left(1 + \frac{1}{m+2} + \frac{1}{(m+2)^2} + \dots + \frac{1}{(m+2)^{n-m-1}}\right)\\
                        &<\frac{1}{(m+1)!} \sum_{k = 0}^\infty (m+2)^{-k}.
    \end{align*}

    Pela \cref{prop:exercício_5}, temos
    \begin{equation*}
        \abs{x_n - x_m} < \frac{1}{(m+1)!}\frac{m + 2}{m + 1} < \frac{2}{(m+1)!},
    \end{equation*}
    para \(m > 0\). Assim, podemos tornar \(\abs{x_n - x_m}\) arbitrariamente pequeno ao escolher \(m\) suficientemente grande, isto é, a sequência é de Cauchy em relação à métrica usual.

    Suponhamos por contradição que a sequência converge a algum número racional \(e = \frac{p}{q}\), com \(p\) e \(q\) coprimos. Assim, dado \(\varepsilon > 0\), existe \(N > 0\) tal que
    \begin{equation*}
        n > N \implies \abs{e - x_n} < \varepsilon.
    \end{equation*}
    Pela desigualdade triangular, temos
    \begin{align*}
        \abs{e - x_m} &\leq \abs{e - x_n} + \abs{x_n - x_m}\\
                      &< \varepsilon + \frac{2}{(m+1)!}
    \end{align*}
    para \(m > 0\) e \(n > N\). Como \(\varepsilon\) é arbitrário, temos
    \begin{equation*}
        \abs{e - x_m} \leq \frac{2}{(m+1)!},
    \end{equation*}
    para \(m > 0\). Como a sequência é estritamente crescente, temos \(e > x_m\), logo
    \begin{equation*}
        x_m < e \leq x_m + \frac{2}{(m+1)!}.
    \end{equation*}
    Em particular, tomemos \(m = 2,\) então
    \begin{equation*}
        \frac52 < e \leq \frac{17}{6},
    \end{equation*}
    portanto \(2 < e < 3\), isto é, \(q \geq 2\), caso contrário \(e\) seria um inteiro entre inteiros consecutivos.

    Podemos tomar \(m = q\), de modo que
    \begin{align*}
        x_q < \frac{p}{q} \leq x_q + \frac{2}{(q+1)!} &\implies q!x_q < p(q-1)! \leq q!x_q + \frac{2}{(q+1)}\\
                                                      &\implies \sum_{k = 0}^{q} \frac{q!}{k!} < p(q-1)! < \sum_{k = 0}^{q} \frac{q!}{k!} + 1,
    \end{align*}
    já que
    \begin{equation*}
        q \geq 2 \implies \frac{2}{q+1} < 1.
    \end{equation*}
    Notemos que \(\frac{q!}{k!} \in \mathbb{N}\) para todo \(k \in \set{0, 1, \dots, q}\), isto é, \(p(q - 1)!\) é um número natural entre inteiros consecutivos. Essa contradição mostra que \(e \notin \mathbb{Q}.\) Desse modo, a sequência não é convergente nos racionais com a métrica usual.
\end{proof}

\section*{Exercício 7}
\begin{proposition}{Sequência de Cauchy em \((\mathcal{C}([0,1]), d_1)\)}{sequência_funções}
    A sequência de funções \(f : \mathbb{N} \to \mathcal{C}([0,1])\) com \(f_0(x) = f_1(x) = f_2(x) = 1\) e
    \begin{equation*}
        f_n(x) = \begin{cases}
            0, & \text{se }x \in \left[0, \frac12 - \frac1n\right]\\
            n \left(x - \frac12 + \frac1n\right), & \text{se }x \in \left(\frac12 - \frac1n, \frac12\right)\\
            1, & \text{se }x \in \left[\frac12, 1\right]
        \end{cases}
    \end{equation*}
    para \(n > 2\) é de Cauchy no espaço métrico \((\mathcal{C}([0,1]), d_1),\) onde a métrica \(d_1\) está definida na \cref{prop:métrica_d1}.
\end{proposition}
\begin{proof}
    Claramente \(f_0, f_1, f_2 \in \mathcal{C}([0,1])\). Para \(n > 2\), segue que \(f_n\) é contínua nos intervalos \([0,\frac12 - \frac1n)\), \((\frac12-\frac1n, \frac12)\) e \((\frac12, 1])\), uma vez que nestes intervalos a função é definida por funções contínuas, portanto resta ver se a aplicação é contínua nos pontos \(\frac12-\frac1n\) e em \(\frac12\). Temos
    \begin{equation*}
        \lim_{x\to \left(\frac12-\frac1n\right)^-} f_n(x) = 0 = f_n\left(\frac12-\frac1n\right),
    \end{equation*}
    \begin{equation*}
        \lim_{x\to \left(\frac12-\frac1n\right)^+} f_n(x) = \lim_{x\to\left(\frac12-\frac1n\right)^+} n\left(x-\frac12+\frac1n\right) = 0 = f_n\left(\frac12-\frac1n\right),
    \end{equation*}
    \begin{equation*}
        \lim_{x\to \frac12^+} f_n(x) = 1 = f_n\left(\frac12\right),
    \end{equation*}
    e
    \begin{equation*}
        \lim_{x\to \frac12+} f_n(x) = \lim_{x\to\frac12+} n\left(x-\frac12+\frac1n\right) = 1 = f_n\left(\frac12\right)
    \end{equation*}
    portanto \(\lim_{x\to\frac12} f_n(x) = f_n\left(\frac12\right)\) e \(\lim_{x\to\frac12} f_n(x) = f_n\left(\frac12\right)\), isto é, \(f_n\) é contínua em \([0,1]\). Logo, \(f_n \in \mathcal{C}([0,1])\) para todo \(n \in \mathbb{N}\).

    Para \(n > m > 2\) temos
    % \begin{equation*}
    %     f_n(x) - f_m(x) = \begin{cases}
    %         0, &\text{se }x \in \left[0, \frac12-\frac1m\right]\\
    %         -m\left(x - \frac12 + \frac1m\right)&\text{se }x \in \left(\frac12 - \frac1m, \frac12-\frac1n\right]\\
    %         (n-m)\left(x - \frac12\right)&\text{se }x \in \left(\frac12 - \frac1n, \frac12\right]\\
    %         0, &\text{se }x \in \left[\frac12, 1\right]\\
    %     \end{cases},
    % \end{equation*}
    % então
    \begin{equation*}
        \abs{f_n(x) - f_m(x)} = \begin{cases}
            0, &\text{se }x \in \left[0, \frac12-\frac1m\right]\\
            m\left(x - \frac12+ \frac1m\right)&\text{se }x \in \left(\frac12 - \frac1m, \frac12-\frac1n\right]\\
            (m-n)\left(x - \frac12\right)&\text{se }x \in \left(\frac12 - \frac1n, \frac12\right)\\
            0, &\text{se }x \in \left[\frac12, 1\right]\\
        \end{cases}.
    \end{equation*}
    Assim, temos
    \begin{align*}
        d_1(f_n, f_m) &= \int_0^1\dli{x} \abs{f_n(x) - f_m(x)} \\
                      &= \int_{\frac12-\frac1m}^{\frac12 - \frac1n}\dli{x} m\left(x - \frac12 + \frac1m\right) + \int_{\frac12-\frac1n}^{\frac12} \dli{x} (m-n)\left(x - \frac12\right)\\
                      &= \int_{-\frac1m}^{-\frac1n}\dli{u} m\left(u + \frac1m\right) + \int_{-\frac1n}^{0} \dli{u} (m-n)u\\
                      &= \frac{m}{2}\left(\frac{1}{n^2} - \frac{1}{m^2}\right) - \frac1n + \frac1m - \frac{m-n}{2n^2}\\
                      &= \frac{1}{2m}-\frac{1}{2n},
    \end{align*}
    donde segue
    \begin{equation*}
        d_1(f_n, f_m) = \frac12\abs*{\frac1m - \frac1n}
    \end{equation*}
    para todos \(n,m > 2\).

    Dado \(\varepsilon > 0\), para \(n,m > \frac{1}{\varepsilon}\) temos
    \begin{align*}
        d_1(f_n, f_m) &= \frac12\abs*{\frac1m - \frac1n}\\
                      &\leq \frac12\left(\abs*{\frac1{m}} + \abs*{\frac{1}{n}}\right)\\
                      &< \varepsilon,
    \end{align*}
    portanto a sequência é de Cauchy em \((\mathcal{C}([0,1]), d_1)\).
\end{proof}

Para que esta sequência de funções seja convergente, deve existir uma função \(f : [0,1] \to \mathbb{R}\) tal que \(f((\frac12, 1]) = \set{1}\) e \(f([0,\frac12) = \set{0}\), caso contrário \(\lim_{n\to\infty} d_1(f_n, f) \neq 0\). Mostraremos na \cref{prop:exercício_7} que este tipo de função satisfaz a condição de convergência e que não é contínua, concluindo que o espaço métrico \((\mathcal{C}([0,1]), d_1)\) não é completo.

\begin{proposition}{O espaço métrico \((\mathcal{C}([0,1]), d_1)\) não é completo}{exercício_7}
    A sequência de funções da \cref{prop:sequência_funções} converge para a função \(\varphi : [0,1] \to \mathbb{R}\) definida por
    \begin{equation*}
        \varphi(x) = \begin{cases}
            0, &\text{se }x \in \left[0,\frac12\right)\\
            1, &\text{se }x \in \left[\frac12,1\right],
        \end{cases}
    \end{equation*}
    que não é uma função contínua.
\end{proposition}
\begin{proof}
    Para \(n > 2\) temos
    \begin{equation*}
        \abs{f_n(x) - \varphi(x)} = \begin{cases}
            0, &\text{se }x \in \left[0, \frac12-\frac1n\right]\\
            n\left(x - \frac12\right) + 1&\text{se }x \in \left(\frac12 - \frac1n, \frac12\right)\\
            0, &\text{se }x \in \left[\frac12, 1\right]\\
        \end{cases}.
    \end{equation*}
    Assim, temos
    \begin{align*}
        d_1(f_n, \varphi) &= \int_0^1\dli{x} \abs{f_n(x) - \varphi(x)} \\
                          &= \int_{\frac12-\frac1n}^{\frac12}\dli{x} \left[n\left(x - \frac12\right) + 1\right]\\
                          &= \frac1{2n}.
    \end{align*}

    Dado \(\varepsilon > 0\), temos
    \begin{equation*}
        n > \frac{1}{2\varepsilon} \implies d_1(f_n, \varphi) = \frac{1}{2n} < \varepsilon,
    \end{equation*}
    portanto a sequência é convergente a \(\varphi\) em relação à métrica \(d_1\).

    Notemos que
    \begin{equation*}
        \lim_{x \to \frac12^+} \varphi(x) = 1\quad\text{e}\quad\lim_{x\to\frac12^-}\varphi(x) = 0,
    \end{equation*}
    isto é, a função \(\varphi\) não pode ser contínua. Assim, o espaço métrico \((\mathcal{C}([0,1]),d_1)\) não é completo.
\end{proof}

\section*{Exercício 8}
\begin{proposition}{Uma métrica define outra}{exercício_8}
    Seja \((M, d)\) um espaço métrico, então
    \begin{align*}
        d_0 : M \times M &\to [0, \infty)\\
                   (x,y) &\mapsto \frac{d(x,y)}{1+d(x,y)}
    \end{align*}
    é uma métrica em \(M\).
\end{proposition}
\begin{proof}
    Por \(d\) ser uma métrica, temos
    \begin{align*}
        d_0(x,y) = 0 &\iff d(x,y) = 0\\
                     &\iff x = y,
    \end{align*}
    e
    \begin{equation*}
        d_0(y,x) = \frac{d(y,x)}{1 + d(y,x)} = \frac{d(x,y)}{1+d(x,y)} = d_0(x,y),
    \end{equation*}
    portanto resta mostrar que \(d_0\) satisfaz a desigualdade triangular.

    Consideremos a aplicação
    \begin{align*}
        f : [0, \infty) &\to [0, \infty)\\
                    \xi &\mapsto \frac{\xi}{1+\xi}.
    \end{align*}
    Temos
    \begin{equation*}
        f'(\xi) = \frac{1}{(1+\xi)^2}
    \end{equation*}
    para todo \(\xi \in [0,\infty)\). Isto é, \(f'(\xi) > 0\) em todo o seu domínio, portanto é uma função crescente.

    Desse modo, como \(f\) mantém a relação de ordem e \(d_0 = f \circ d\), segue que
    \begin{align*}
        d(x,y) \leq d(x,z) + d(z, y) &\implies (f\circ d)(x,y) \leq f(d(x,z) + d(z,y))\\
                                     &\implies d_0(x,y) \leq \frac{d(x,z) + d(z,y)}{1 + d(x,z) + d(z,y)},
    \end{align*}
    para todo \(x,y,z \in M\). Notemos que
    \begin{align*}
        0 \leq d(x,z) \leq d(x,z) + d(z,y) &\implies 1 \leq 1 + d(x,z) \leq 1 + d(x,z) + d(z,y)\\
                                           &\implies \frac{1}{1+d(x,z) + d(z,y)} \leq \frac{1}{1+d(x,z)} \leq 1\\
                                           &\implies \frac{d(x,z)}{1+d(x,z)+d(y,z)} \leq d_0(x,z) \leq d(x,z),
    \end{align*}
    e analogamente
    \begin{equation*}
        0 \leq d(y,z) \leq d(x,z) + d(z,y) \implies \frac{d(z,y)}{1+d(x,z)+d(y,z)} \leq d_0(z,y) \leq d(z,y).
    \end{equation*}
    Portanto
    \begin{equation*}
        \frac{d(x,z) + d(z,y)}{1 + d(x,z) + d(z,y)} \leq d_0(x,z) + d_0(z,y),
    \end{equation*}
    donde segue
    \begin{equation*}
        d_0(x,y) \leq d_0(x,z) + d_0(z,y),
    \end{equation*}
    isto é, \(d_0\) satisfaz a desigualdade triangular. Logo, \(d_0\) é uma métrica em \(M\).
\end{proof}

\section*{Exercício 9}

\begin{proposition}{O intervalo aberto \((a,b)\) não é um espaço métrico completo em relação à métrica usual}{aberto}
    Consideremos o intervalo aberto não vazio \(\mathring{A} = (a,b) \subset \mathbb{R}\) e a métrica usual
    \begin{align*}
        d : \mathbb{R} \times \mathbb{R} &\to [0, \infty)\\
                                   (x,y) &\mapsto \abs{x - y}.
    \end{align*}
    O espaço métrico \((\mathring{A}, d)\) não é completo.
\end{proposition}
\begin{proof}
    Consideremos a sequência \(s : \mathbb{N} \to \mathring{A}\) definida por \(s_0 = s_1 = \frac{a+b}{2}\) e
    \begin{equation*}
        s_n = a + \frac{b - a}{n}
    \end{equation*}
    para \(n \geq 2\).

    Dado \(\varepsilon > 0\), podemos tomar \(M = \frac{2(b-a)}{\varepsilon}\) tal que para todos \(n,m > M\) vale
    \begin{align*}
        d(s_m, s_n) &= (b - a) \abs*{\frac{1}{m} - \frac{1}{n}}\\
                    &\leq \frac{b - a}{m} + \frac{b - a}{n}\\
                    &< \frac{2(b - a)}{M} = \varepsilon,
    \end{align*}
    isto é, \(s_n\) é uma sequência de Cauchy em relação à métrica usual.


    Dado \(\varepsilon > 0\) podemos tomar \(N = \frac{b-a}{\varepsilon}\) tal que para todo \(n > N\) temos
    \begin{equation*}
        d(a, s_n) = \frac{b - a}{n} < \frac{b-a}{N} = \varepsilon,
    \end{equation*}
    isto é, \(s_n\) converge a \(a \notin \mathring{A}\) em relação à métrica usual.

    Encontramos uma sequência de Cauchy em \((\mathring{A}, d)\) que não converge neste espaço métrico, portanto \((\mathring{A},d)\) não é completo.
\end{proof}
\begin{corollary}
    O intervalo \((0,1)\) não é completo em relação à métrica usual.
\end{corollary}

\begin{lemma}{Toda sequência de números reais tem uma subsequência monotônica}{monotônica}
    Seja \(s: \mathbb{N} \to X \subset \mathbb{R}\) uma sequência de números reais. Então existe uma sequência crescente de números naturais \(n : \mathbb{N} \to \mathbb{N}\) tal que a subsequência \(s\circ n\) é monotônica, isto é, ou é monotônica decrescente \(s_{n_{i}} \leq s_{n_{j}}\) para todo \(i > j\), ou é monotônica crescente \(s_{n_{i}} \geq s_{n_{j}}\) para todo \(i > j\).
\end{lemma}
\begin{proof}
    Consideremos o conjunto
    \begin{equation*}
        S = \set{k \in \mathbb{N} : m > k \implies s_k \geq s_m}
    \end{equation*}
    dos índices dos elementos da sequência que são maiores que os elementos subsequentes.

    Se \(S\) é infinito, então existe uma sequência crescente de números naturais \(n : \mathbb{N} \to S \subset \mathbb{N}\) definida por uma enumeração dos elementos de \(S\) tal que a subsequência \(s \circ n\) é monotônica decrescente. De fato, sejam \(n_i, n_j \in S\) com \(i < j \implies n_i < n_j\), então
    \begin{equation*}
        n_i \in S \implies s_{n_i} \geq s_{n_j},
    \end{equation*}
    isto é, \(s\circ n\) é monotônica decrescente.

    Se \(S\) não é infinito, então o seu complemento \(T = \mathbb{N} \smallsetminus S\) é infinito. Notemos que
    \begin{equation*}
        T = \set{k \in \mathbb{N} : \exists m > k\text{ tal que }s_k < s_m}.
    \end{equation*}
    Dado \(k \in T\), temos que o conjunto
    \begin{equation*}
        M_k = \set{m > k : s_m > s_a},
    \end{equation*}
    é não vazio, pela definição de \(T\). Assim, podemos definir a sequência \(n : \mathbb{N} \to T \subset \mathbb{N}\) com
    \begin{equation*}
        T \ni n_0 = \begin{cases}
            0 &\text{ se }S=\emptyset\\
            1 + \max{S}&\text{ se }S \neq \emptyset
        \end{cases}
    \end{equation*}
    e
    \begin{equation*}
        n_{i+1} = \min{M_{n_i}}.
    \end{equation*}
    Desse modo, temos
    \begin{equation*}
        i < j \implies s_{n_{i}} < s_{n_{i+1}} < \dots < s_{n_{j-1}} < s_{n_j},
    \end{equation*}
    isto é, \(s\circ n\) é uma subsequência monotônica crescente.
\end{proof}

\begin{lemma}{Subsequência convergente de uma sequência de Cauchy}{subsequência}
    Seja \((X,d)\) um espaço métrico e seja \family{x_n}{n\in \mathbb{N}} uma sequência de Cauchy em \((X,d)\). Se existe uma sequência crescente \family{n_j}{j\in\mathbb{N}} de números naturais tal que a subsequência \family{x_{n_j}}{j\in\mathbb{N}} é convergente em \((X,d)\), então \(x_n\) converge em \((X,d)\).
\end{lemma}
\begin{proof}
    Dado \(\varepsilon > 0,\) existe \(N > 0\) tal que para todo \(m,n > N\) vale
    \begin{equation*}
        d(x_n, x_m) < \frac12 \varepsilon,
    \end{equation*}
    já que a sequência é de Cauchy em relação à métrica \(d\).

    Seja \(x \in X\) o ponto ao qual a subsequência converge. Então dado \(\varepsilon > 0,\) existe \(J > 0\) tal que para todo \(n_j > J\) vale
    \begin{equation*}
        d(x_{n_j}, x) < \frac12 \varepsilon.
    \end{equation*}

    Seja \(M = \max\set{N,J}\), então para todos \(m, n_j > M\) segue que
    \begin{align*}
        d(x_m, x) &\leq d(x_m, x_{n_j}) + d(x_{n_j}, x)\\
                  &< \varepsilon,
    \end{align*}
    isto é, a sequência de Cauchy converge para \(x \in X\) em relação à métrica \(d\).
\end{proof}

\begin{proposition}{O intervalo fechado \([a,b]\) é um espaço métrico completo em relação à métrica usual}{fechado}
    Consideremos o intervalo fechado \(\bar{A} = [a,b] \subset \mathbb{R}\) e a métrica usual \(d\). O espaço métrico \((\bar{A}, d)\) é completo.
\end{proposition}
\begin{proof}
    Seja \(s : \mathbb{N} \to [a,b] \subset \mathbb{R}\) uma sequência de Cauchy em relação à métrica usual. Pelo \cref{lem:monotônica}, existe uma sequência crescente de números naturais \(n : \mathbb{N} \to \mathbb{N}\) tais que a subsequência \(x = s \circ n\) é monotônica. Notemos que esta subsequência também é de Cauchy: dado \(\varepsilon > 0\), existe \(N_{\varepsilon} > 0\) tal que
    \begin{equation*}
        \ell,k > N_{\varepsilon} \implies d(s_\ell, s_k) < \varepsilon
    \end{equation*}
    então tomando \(M = \min\set{m \in \mathbb{N} : n_m > N_{\varepsilon})}\), segue que
    \begin{align*}
        i,j > M &\implies d(s_{n_i},s_{n_j}) < \varepsilon\\
                &\implies d(x_i, x_j) < \varepsilon,
    \end{align*}
    logo \(x\) é de Cauchy.

    Suponhamos que a subsequência é monotônica decrescente. Como uma sequência de Cauchy em \((\mathbb{R}, d)\), segue que existe \(\xi \in \mathbb{R}\) tal que \(x\) converge a \(\xi\) em relação a este espaço métrico completo. Assim, dado \(\varepsilon > 0\), existe \(N_{\varepsilon} > 0\) tal que
    \begin{equation*}
        i > N_{\varepsilon} \implies d(x_i, \xi) < \varepsilon.
    \end{equation*}
    Suponhamos por contradição que \(\xi \notin [a,b]\), então \(\xi < a\). Tomemos \(\varepsilon = \frac{a - \xi}2\), então existe \(N > 0\) tal que
    \begin{align*}
        i > N &\implies - \frac{a - \xi}2 < x_i - \xi < \frac{a - \xi}2\\
              &\implies \frac{3\xi - a}2 < x_i < \frac{a + \xi}2.
    \end{align*}
    Notemos entretanto que \(a + \xi < 2a\), portanto devemos ter \(x_i < a\). Esta contradição mostra que \(\xi \in [a,b]\), portanto \(x\) converge para algum valor de \([a,b]\). Pelo \cref{lem:subsequência}, a sequência de Cauchy \(s\) converge em \(([a,b],d)\).

    Um argumento análogo pode ser feito para o caso em que a subsequência é monotônica crescente. Neste caso, \(\xi > b\), então podemos tomar \(\varepsilon = \frac{\xi - b}{2}\), então existe \(N > 0\) tal que
    \begin{align*}
        i > N &\implies - \frac{\xi - b}2 < x_i - \xi < \frac{\xi - b}2\\
              &\implies \frac{b + \xi}2 < x_i < \frac{3\xi - b}2.
    \end{align*}
    Então deveríamos ter \(x_i > b\), o que não pode acontecer. Assim, concluímos que a sequência de Cauchy \(s\) deve convergir em \(([a,b],d)\).
\end{proof}
\begin{corollary}
    O intervalo \([0,1]\) é um espaço métrico completo em relação à métrica usual.
\end{corollary}

\begin{lemma}{Desigualdade triangular inversa}{desigualdade}
    Seja \((X,d)\) um espaço métrico, então
    \begin{equation*}
        d(x,y) \geq \abs{d(x,z) - d(z,y)}
    \end{equation*}
    para todo \(x,y,z \in X\).
\end{lemma}
\begin{proof}
    Pela desigualdade triangular temos
    \begin{equation*}
        d(x,z) \leq d(x,y) + d(y,z) \implies d(x,z) - d(y,z) \leq d(x,y).
    \end{equation*}
    Suponhamos que \(d(x,z) - d(y,z) \geq 0\), então
    \begin{equation*}
        d(x,y) \geq \abs{d(x,z) - d(y,z)}.
    \end{equation*}
    Suponhamos agora que \(d(x,z) - d(y,z) < 0\), então pela desigualdade triangular temos
    \begin{align*}
        d(y,z) \leq d(y,x) - d(x,z) &\implies d(y,z) - d(x,z) \leq d(x,y)\\
                                    &\implies \abs{d(x,z) - d(y,z)} \leq d(x,y).
    \end{align*}
    Assim, mostramos que
    \begin{equation*}
        d(x,y) \geq \abs{d(x,z) - d(y,z)}
    \end{equation*}
    para todo \(x,y,z \in X\).
\end{proof}

\begin{lemma}{Sequência de Cauchy de números reais é limitada}{limitada}
    Seja \(s : \mathbb{N} \to X \subset \mathbb{R}\) uma sequência de Cauchy em relação à métrica usual. Então existe \(M > 0\) tal que
    \begin{equation*}
        n \in \mathbb{N} \implies \abs{s_n} \leq M,
    \end{equation*}
    isto é, a sequência é limitada.
\end{lemma}
\begin{proof}
    Dado \(\varepsilon > 0,\) existe \(N_{\varepsilon} > 0\) tal que
    \begin{equation*}
        n, m > N_{\varepsilon} \implies \abs{s_n - s_m} < \varepsilon.
    \end{equation*}
    Em particular, tomamos \(\varepsilon = 1\) e \(n_0\) o primeiro natural tal que
    \begin{equation*}
        n > n_0 \implies \abs{s_n - s_{n_0}} < 1
    \end{equation*}

    Pelo \cref{lem:desigualdade}, temos que
    \begin{equation*}
        \abs{s_n - s_m} \geq \abs{\abs{s_n} - \abs{s_m}} \implies -\abs{s_n - s_m}\leq \abs{s_n} - \abs{s_m} \leq \abs{s_n - s_m}
    \end{equation*}
    portanto
    \begin{align*}
        n > n_0 &\implies \abs{s_n} - \abs{s_{n_0}} < 1\\
                &\implies \abs{s_n} < 1 + \abs{s_{n_0}}.
    \end{align*}

    Assim, definimos
    \begin{equation*}
        M = \max\set*{\abs{s_0}, \abs{s_1}, \dots, \abs{s_{n_0-1}}, \abs{s_{n_0}}, \abs{s_{n_0}}+ 1}
    \end{equation*}
    de forma que
    \begin{equation*}
        \abs{s_n} \leq M
    \end{equation*}
    para todo \(n \in \mathbb{N}\).
\end{proof}

\begin{proposition}{O intervalo \([a,\infty)\) é um espaço métrico completo em relação à métrica usual}{intervalo1inf}
    O espaço métrico \(([a,\infty), d)\), em que \(d\) é a métrica usual, é completo.
\end{proposition}
\begin{proof}
    Seja \(s : \mathbb{N} \to [a,\infty)\) uma sequência de Cauchy em relação à métrica usual. Pelo \cref{lem:limitada}, existe \(M > 0\) tal que \(\abs{s_n} \leq M\). Desse modo, a imagem da sequência deve estar contida no intervalo fechado \([a, M]\). Isto é, \(s\) é uma sequência de Cauchy no espaço métrico \([a, M]\), que é completo em relação à métrica usual pela \cref{prop:fechado}, logo existe \(\sigma \in [a,M]\) ao qual \(s\) converge. Como \([a, M] \subset [a, \infty)\), então \(\sigma \in [a, \infty)\). Assim, \(([a,\infty), d)\) é um espaço métrico completo em relação à métrica usual.
\end{proof}
\begin{corollary}
    O intervalo \((-\infty, a]\) é completo em relação à métrica usual.
\end{corollary}
\begin{corollary}
    O intervalo \([1, \infty)\) é completo em relação à métrica usual.
\end{corollary}

\begin{proposition}{Métrica no intervalo \([1, \infty)\)}{métrica_recíproca}
    O espaço métrico \(([1,\infty), d_I)\) não é completo, onde a métrica \(d_I\) é a aplicação
    \begin{align*}
        d_I : [1,\infty) \times [1, \infty) &\to [0, \infty)\\
                                      (x,y) &\mapsto \abs*{\frac1x - \frac1y}.
    \end{align*}
\end{proposition}
\begin{proof}
    Primeiro mostramos que \(d_I\) é de fato uma métrica em \([1,\infty)\). Para \(x,y \in [1,\infty)\), temos
    \begin{align*}
        x = y &\iff \frac1x = \frac1y\\
              &\iff \frac{1}{x} - \frac1y = 0\\
              &\iff d_I(x,y) = 0
    \end{align*}
    e
    \begin{equation*}
        d_I(y,x) = \abs*{\frac1y - \frac1x} = \abs*{\frac1x - \frac1y} = d_I(x,y).
    \end{equation*}
    Para todo \(x,y,z \in [1,\infty)\),
    \begin{align*}
        d_I(x,y) &= \abs*{\frac1x - \frac1z + \frac1z - \frac1y}\\
                 &\leq \abs*{\frac1x - \frac1z} + \abs*{\frac1z - \frac1y},
    \end{align*}
    isto é,
    \begin{equation*}
        d_I(x,y) \leq d_I(x,z) + d_I(z,y).
    \end{equation*}
    Assim, \(([1,\infty), d_I)\) é um espaço métrico.

    Consideremos a sequência \(s : \mathbb{N} \to [1,\infty)\) definida por \(s_0 = 1\) e \(s_n = n\) para \(n \geq 1\). Dado \(\varepsilon > 0\), tomemos \(N = \frac{2}{\varepsilon}\), então para todos \(n,m > N\) vale
    \begin{align*}
        d_I(s_n, s_m) &= \abs*{\frac1{n} - \frac1{m}}\\
                      &\leq \frac{1}{n} + \frac{1}{m}\\
                      &< \frac{2}{N} = \varepsilon,
    \end{align*}
    isto é, \(s\) é uma sequência de Cauchy em \(([1,\infty), d_I)\).

    Entretanto, \(s\) não converge neste espaço métrico. De fato, suponhamos por contradição que existe \(\sigma \in [1,\infty)\) ao qual a sequência converge. Neste caso, dado \(\varepsilon > 0\), existe \(M > 0\) tal que para todo \(n > M\) vale \(d_I(s_n, \sigma) < \varepsilon\).
    Consideremos \(\varepsilon = \frac{1}{2\sigma} \in (0,1]\), então
    \begin{align*}
        d_I(s_n, \sigma) < \varepsilon &\implies \abs*{\frac1{s_n} - \frac1\sigma} < \frac{1}{2\sigma}\\
                                       &\implies -\frac{1}{2\sigma} < \frac1{s_n} - \frac{1}\sigma < \frac1{2\sigma}\\
                                       &\implies \frac{1}{2\sigma} < \frac1{s_n} < \frac{3}{2\sigma}\\
                                       &\implies \frac{2\sigma}{3} < s_n < 2\sigma.
    \end{align*}
    para todo \(n > M\). Isto é, \(2\sigma\) deve ser maior do que qualquer número natural maior do que \(M\), o que contradiz a propriedade arquimediana dos números reais. De fato, se \(k \in \mathbb{N}\) é tal que \(k > M\) e \(s_{k} < 2\sigma\), então existe \(\ell \in \mathbb{N}\) tal que \(\ell s_{k} > 2\sigma\), ou seja, \(\ell k > M\) e \(s_{\ell k} > 2\sigma\). Dessa forma, não pode existir \(\sigma \in [1,\infty)\) ao qual a sequência de Cauchy \(s\) converge em relação à métrica \(d_I\), portanto este espaço métrico não é completo.
\end{proof}

\end{document}
