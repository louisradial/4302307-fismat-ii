\section*{Exercício 3}
\begin{proposition}{Solução da equação de Airy}{airy}
    A solução geral da equação de Airy,
    \begin{equation*}
        \diff[2]yx - xy(x) = 0
    \end{equation*}
    é a função inteira dada por
    \begin{equation*}
        y(x) = c_0\sum_{k=0}^\infty\frac{(3k + 1)!!!}{(3k+1)!}x^{3k} + c_1\sum_{k = 0}^\infty \frac{(3k+2)!!!}{(3k+2)!}x^{3k+1},\quad\text{com }c_0,c_1 \in \mathbb{C},
    \end{equation*}
    onde \(n!!! = n(n-3)(n-6)\dots(n_3 + 3) n_3\), com \(n_3\) dado pelo resto natural da divisão de \(n\) por três e \(0!!! = 1\).
\end{proposition}
\begin{proof}
    Notemos que a equação é do tipo
    \begin{equation*}
        \diff[2]yx + p(x) \diff{y}{x} + q(x)y(x) = 0,
    \end{equation*}
    com \(p(x) = 0\) e \(q(x) = -x\). Como \(p\) e \(q\) são funções inteiras, então uma solução do tipo \(y(x) = \sum_{k = 0}^\infty c_k x^k\) converge em todo o plano complexo, isto é, a solução é uma função inteira.

    Para uma solução deste tipo temos
    \begin{equation*}
        \diff{y}{x} = \sum_{k = 0}^\infty kc_kx^{k-1}\quad\text{e}\quad\diff[2]yx = \sum_{k = 0}^\infty k(k-1)c_kx^{k-2},
    \end{equation*}
    portanto obtemos
    \begin{equation*}
        \sum_{k = 2}^{\infty} k(k-1)c_kx^{k-2} - \sum_{k = 0}^\infty c_kx^{k+1} = 0
    \end{equation*}
    ao substituir na equação diferencial. Manipulando os índices das séries, temos
    \begin{equation*}
        2c_2 + \sum_{k = 0}^\infty\left[(k+3)(k+2)c_{k+3} - c_k\right]x^{k+1} = 0,
    \end{equation*}
    portanto obtemos \(c_2 = 0\) e a relação de recorrência
    \begin{equation*}
        c_{k+3} = \frac{c_k}{(k+3)(k+2)}
    \end{equation*}
    para todo \(k \in \mathbb{N}\). Dessa forma, é fácil ver que para todo \(\ell \in \mathbb{N}\), temos \(c_{3\ell+2} = 0\).

    Mostremos por indução que
    \begin{equation*}
        c_{3\ell} = \frac{(3\ell+1)!!!}{(3\ell+1)!}c_0\quad\text{e}\quad c_{3\ell + 1} = \frac{(3\ell + 2)!!!}{(3\ell+2)!}c_1
    \end{equation*}
    para todo \(\ell \in \mathbb{N}\). Para \(\ell = 0\) temos
    \begin{equation*}
        \frac{1!!!}{1!}c_0 = c_0\quad\text{e}\quad\frac{2!!!}{2!}c_1 = c_1,
    \end{equation*}
    portanto as expressões valem neste caso. Suponhamos que as igualdades sejam satisfeitas para algum \(m \in \mathbb{N}\), então das relações de recorrência temos
    \begin{align*}
        c_{3m + 3} &= \frac{c_{3m}}{(3m+3)(3m+2)} \\&= \frac{(3m + 1)!!!}{(3m+3)(3m+2)(3m + 1)!}c_0 \\&= \frac{(3m+4)(3m+1)!!!}{(3m+4)!}c_0 \\&= \frac{(3m+4)!!!}{(3m+4)!}c_0
    \end{align*}
    e semelhantemente
    \begin{align*}
        c_{3m + 4} &= \frac{c_{3m+1}}{(3m+4)(3m+3)} \\&= \frac{(3m+5)(3m + 2)!!!}{(3m+5)(3m+4)(3m+3)(3m + 2)!}c_1 \\&= \frac{(3m+5)!!!}{(3m+5)!}c_1,
    \end{align*}
    isto é, as identidades são satisfeitas por \(m + 1\). Pelo princípio da indução finita, segue que as expressões são válidas para todo \(\ell \in \mathbb{N}\).

    Logo, a solução geral da equação de Airy é a função inteira
    \begin{equation*}
        y(x) = c_0 \sum_{\ell = 0}^\infty\frac{(3\ell + 1)!!!}{(3\ell + 1)!}x^{3\ell} + c_1 \sum_{\ell = 0}^\infty\frac{(3\ell + 2)!!!}{(3\ell + 2)!}x^{3\ell + 1},
    \end{equation*}
    com \(c_0, c_1 \in \mathbb{N}\).
\end{proof}
