\section*{Exercício 1}
\begin{proposition}{Solução geral para o oscilador harmônico amortecido}{exercício01a}
    Sejam \(a, b, c\) constantes reais com \(a \neq 0\) e \(b^2 \neq 4ac\). A equação diferencial
    \begin{equation*}
        a \ddot{x}(t) + b \dot{x}(t) + cx(t) = 0
    \end{equation*}
    admite uma solução do tipo \(x(t) = e^{\lambda t}\), com \(\lambda \in \mathbb{C}\) satisfazendo
    \begin{equation*}
        a\lambda^2 + b \lambda + c = 0,
    \end{equation*}
    isto é,
    \begin{equation*}
        \lambda = \lambda_\pm = \frac{-b \pm \sqrt{b^2 - 4ac}}{2a}.
    \end{equation*}
    A solução geral da equação diferencial é dada por
    \begin{equation*}
        x(t) = \alpha e^{\lambda_+t} + \beta e^{\lambda_-t},
    \end{equation*}
    com \(\alpha, \beta \in \mathbb{C}\) constantes.
\end{proposition}
\begin{proof}
    Mostremos que \(x(t) = e^{\lambda t}\) é solução da equação diferencial. Temos
    \begin{align*}
        \dot{x}(t) &= \lambda e^{\lambda t}& \ddot{x}(t) &= \lambda^2 e^{\lambda t}\\ &= \lambda x(t) & &= \lambda^2 x(t),
    \end{align*}
    portanto ao substituir no lado esquerdo da equação diferencial temos
    \begin{align*}
        a \ddot{x}(t) + b\dot{x}(t) + cx(t) &= a \lambda^2 x(t) + b \lambda x(t) + c x(t)\\
                                            &= \left(a \lambda^2 + b \lambda + c\right)x(t).
    \end{align*}
    Como \(x(t) > 0\) para todo \(t \in \mathbb{R}\), segue que o ansatz é solução se e somente se
    \begin{equation*}
        a \lambda^2 + b \lambda + c = 0,
    \end{equation*}
    como desejado.

    Como a equação diferencial ordinária é linear e de segunda ordem, então a solução geral é dada por
    \begin{equation*}
        x(t) = \alpha x_1(t) + \beta x_2(t),
    \end{equation*}
    com \(\set{x_1, x_2}\) um conjunto linearmente independente de soluções da equação diferencial. Notemos que \(\set{e^{\lambda_+ t}, e^{\lambda_- t}}\) é linearmente independente:
    \begin{equation*}
        \mu e^{\lambda_+ t} + \nu e^{\lambda_- t} = 0 \implies \mu = \nu = 0,
    \end{equation*}
    portanto a solução geral da equação diferencial estudada é
    \begin{equation*}
        x(t) = \alpha e^{\lambda_+ t} + \beta e^{\lambda_- t},
    \end{equation*}
    como queríamos mostrar.
\end{proof}

\begin{proposition}{Solução para o oscilador harmônico amortecido com condições iniciais}{exercício01c}
    Sejam \(a,b,c\) como antes. O problema de valor inicial
    \begin{equation*}
        a \ddot{x}(t) + b \dot{x}(t) + cx(t) = 0
    \end{equation*}
    com \(x(0) = x_0\) e \(\dot{x}(0) = v_0\) tem como solução
    \begin{equation*}
        x(t) = \left[\frac{v_0 - \lambda_- x_0}{\lambda_+ - \lambda_-}\right]e^{\lambda_+ t} + \left[\frac{\lambda_+ x_0 - v_0}{\lambda_+ - \lambda_-}\right]e^{\lambda_- t}.
    \end{equation*}
    Ainda, definindo \(\gamma = \frac{b}{2a}\) e \(\sigma = \frac{\sqrt{b^2 - 4ac}}{2a}\), temos
    \begin{equation*}
        x(t) = e^{-\gamma t} \left[x_0\cosh(\sigma t) + \left(\frac{\gamma x_0 + v_0}{\sigma}\right)\sinh(\sigma t)\right].
    \end{equation*}
\end{proposition}
\begin{proof}
    Da \cref{prop:exercício01a}, temos
    \begin{equation*}
        x(t) = \alpha e^{\lambda_+ t} + \beta e^{\lambda_- t} \implies x(0) = \alpha + \beta
    \end{equation*}
    e
    \begin{equation*}
        \dot{x}(t) = \alpha \lambda_+ e^{\lambda_+ t} + \beta \lambda_- e^{\lambda_- t} \implies \dot{x}(0) = \lambda_+\alpha + \lambda_-\beta.
    \end{equation*}
    Dessa forma, das condições iniciais, obtemos o sistema de equações lineares
    \begin{equation*}
       \begin{pmatrix}
           1 & 1\\\lambda_+&\lambda_-
       \end{pmatrix}
       \begin{bmatrix}
           \alpha \\ \beta
       \end{bmatrix}
       =
       \begin{bmatrix}
           x_0\\
           v_0
       \end{bmatrix},
    \end{equation*}
    cujas soluções são
    \begin{equation*}
        \alpha = \frac{v_0 - x_0 \lambda_-}{\lambda_+ - \lambda_-}\quad\text{e}\quad \beta = \frac{\lambda_+ x_0 - v_0}{\lambda_+ - \lambda_-}.
    \end{equation*}
    Isto é, a solução do problema de valor inicial é
    \begin{align*}
        x(t) &= \left[\frac{v_0 - \lambda_- x_0}{\lambda_+ - \lambda_-}\right]e^{\lambda_+ t} + \left[\frac{\lambda_+ x_0 - v_0}{\lambda_+ - \lambda_-}\right]e^{\lambda_- t}\\
             &= \frac{-\lambda_- e^{\lambda_+ t} + \lambda_+ e^{\lambda_- t}}{\lambda_+ - \lambda_-}x_0 + \frac{e^{\lambda_+ t} - e^{\lambda_-t}}{\lambda_+ - \lambda_-}v_0
    \end{align*}
    como desejado.

    Definimos \(\gamma = \frac{b}{2a}\) e \(\sigma = \frac{\sqrt{b^2 - 4ac}}{2a}\), de modo que \(\lambda_\pm = - \gamma \pm \sigma\). Assim, temos
    \begin{align*}
        e^{\gamma t}x(t) &= \frac{(\gamma + \sigma)e^{\sigma t} + (-\gamma + \sigma)e^{-\sigma t}}{2\sigma}x_0 + \frac{e^{\sigma t} - e^{\sigma t}}{2\sigma}v_0\\
                         &= \frac{e^{\sigma t} - e^{-\sigma t}}{2\sigma} \left(\gamma x_0 + v_0\right) + \frac{e^{\sigma t} + e^{-\sigma t}}{2} x_0,
    \end{align*}
    isto é, a solução para o problema de valor inicial pode ser dada por
    \begin{equation*}
        x(t) = e^{-\gamma t} \left[x_0\cosh(\sigma t) + \left(\frac{\gamma x_0 + v_0}{\sigma}\right)\sinh(\sigma t)\right],
    \end{equation*}
    como queríamos demonstrar.
\end{proof}

\begin{figure}[!ht]
    \centering
    \begin{tikzpicture}
        \begin{axis}[
            width=0.9\linewidth,
            height=0.3\textheight,
            xmin=0, xmax=15,
            ymin=-0.3,ymax=0.8,
            domain=0:15,
            samples=500,
            axis lines=middle,
            xlabel={\(t\)},
            ylabel={\(x(t)\)},
            legend pos=north east,
            ytick=\empty,
            xtick=\empty
        ]
        \addplot[thick, Mauve] {0.5*(cos(deg(x)) + 0.3*sin(deg(x)))*exp(-0.3*x)};
            \addlegendentry{\(b^2 - 4ac < 0\)}

                \addplot[thick, Pink] {0.5*(cosh(x) + 1.6*sinh(x))*exp(-1.6*x)};
            \addlegendentry{\(b^2 - 4ac > 0\)}
        \end{axis}
    \end{tikzpicture}
    \caption{Algumas soluções reais para o oscilador harmônico amortecido com velocidade inicial nula.}
    \label{fig:exercício01}
\end{figure}

\begin{proposition}{Oscilador harmônico amortecido: caso supercrítico}{exercício01f}
    Sejam \(a,b,c \in \mathbb{R}\) constantes com \(a \neq 0\) e \(b^2 = 4ac\). A função
    \begin{equation*}
        x(t) = \alpha e^{\lambda t} + \beta t e^{\lambda t}
    \end{equation*}
    com \(\alpha, \beta \in \mathbb{C}\) e \(\lambda = -\frac{b}{2a}\) é solução da equação diferencial
    \begin{equation*}
        a\ddot{x}(t) + b\dot{x}(t) + cx(t) = 0.
    \end{equation*}
\end{proposition}
\begin{proof}
    Verifiquemos diretamente computando as derivadas da solução proposta. Temos
    \begin{align*}
        \dot{x}(t) &= \alpha \lambda e^{\lambda t} + \beta e^{\lambda t} + \beta \lambda t e^{\lambda t}\\
                   &= \lambda x(t) + \beta e^{\lambda t},
    \end{align*}
    portanto
    \begin{align*}
        \ddot{x}(t) &= \lambda \dot{x}(t) + \beta \lambda e^{\lambda t}\\
                    &= \lambda^2 x(t) + 2 \beta \lambda e^{\lambda t}.
    \end{align*}
    Substituindo na equação diferencial, obtemos
    \begin{equation*}
        a \ddot{x}(t) + b\dot{x}(t) + cx(t) = \left(a \lambda^2 + b \lambda + c\right)x(t) + (2a \lambda + b)\beta e^{\lambda t} = 0,
    \end{equation*}
    visto que \(a \lambda^2 + b \lambda + c = 0\) e \(2a \lambda + b = 0\). Isto é, mostramos que \(x(t)\) é de fato solução da equação diferencial.
\end{proof}
