\section*{Exercício 10}
\begin{lemma}{Ponto fixo único de uma função iterada}{ponto_fixo_iteração}
    Seja \(X\) um conjunto não vazio onde está definida a função \(f : X \to X\). Se a iteração \(g = f\circ f\) tem um único ponto fixo, então \(f\) tem um único ponto fixo.
\end{lemma}
\begin{proof}
    Seja \(y \in X\) o único ponto fixo de \(g\). Pela associatividade da composição de funções, temos
    \begin{equation*}
        f(y) = (f\circ g)(y) = (f\circ f)(f(y)) = g(f(y)),
    \end{equation*}
    isto é, \(f(y)\) é um ponto fixo de \(g\), portanto \(y = f(y)\). Assim, \(y\) é ponto fixo de \(f\).

    Suponhamos, por contradição, que \(x \in X\smallsetminus \set{y}\) é um ponto fixo de \(f\). Temos
    \begin{equation*}
        g(x) = f(f(x)) = f(x) = x,
    \end{equation*}
    portanto \(x\) é um outro ponto fixo de \(g\). Esta contradição mostra que \(y\) é o único ponto fixo de \(f\) em \(X\).
\end{proof}

\begin{lemma}{União finita de subespaços métricos completos é um espaço métrico completo}{união_completa}
    Seja \((M,d)\) um espaço métrico e sejam \(M_1, M_2 \subset M\) subconjuntos não vazios. Se \((M_1, \restrict{d}{M_1\times M_1})\) e \((M_2, \restrict{d}{M_2\times M_2})\) são espaços métricos completos, então \((X, \restrict{d}{X\times X})\) é um espaço métrico completo, com \(X = M_1 \cup M_2\).
\end{lemma}
\begin{proof}
    Seja \family{x_n}{n\in \mathbb{N}} uma sequência de Cauchy em \(X\). Então existe \(k \in \set{1,2}\) tal que existem infinitos \(x_n \in M_k\). Assim, existe uma sequência crescente \family{n_j}{j\in \mathbb{N}} de números naturais tal que \(\family{x_{n_j}}{j\in \mathbb{N}} \subset M_k\) é uma subsequência em \((M_k,\restrict{d}{M_k \times M_k})\). Como a sequência é de Cauchy em \((X, \restrict{d}{X\times X})\), segue que dado \(\varepsilon > 0\), existe \(N > 0\) tal que para todo \(n,m > N\) vale
    \begin{equation*}
        \restrict{d}{X\times X}(x_n, x_m) < \varepsilon,
    \end{equation*}
    em particular para \(n_i, n_j > N\) temos
    \begin{equation*}
        \restrict{d}{X\times X}(x_{n_i}, x_{n_j}) = \restrict{d}{M_k \times M_k}(x_{n_i}, x_{n_j}) < \varepsilon,
    \end{equation*}
    então esta subsequência é de Cauchy em \((M_k, \restrict{d}{M_k\times M_k})\), logo convergente neste espaço métrico completo. Pelo \cref{lem:subsequência}, segue que \family{x_n}{n\in \mathbb{N}} converge em \((X, \restrict{d}{X\times X})\), o que conclui a demonstração.
\end{proof}
\begin{proposition}{Constante Omega}{lambert}
    A única solução de
    \begin{equation*}
        x = e^{-x}
    \end{equation*}
    para \(x \in [0,\infty)\) é a constante \(\Omega = W(1) \simeq 0.56714.\)
\end{proposition}
\begin{proof}
    Pela definição da função \(W\) de Lambert, temos
    \begin{equation*}
        \Omega e^\Omega = 1 \implies \Omega = e^{-\Omega},
    \end{equation*}
    portanto resta mostrar que esta equação não possui outra solução em \([0,\infty)\). Alternativamente, mostramos que \(\Omega\) é um ponto fixo da aplicação suave
    \begin{align*}
        f : [0, \infty) &\to [0, \infty)\\
                      t &\mapsto e^{-t}
    \end{align*}
    e desejamos mostrar que \(\Omega\) é o único ponto fixo de \(f\) em \([0,\infty)\).

    Consideremos a aplicação suave
    \begin{align*}
        g : [0, \infty) &\to [0,\infty)\\
                      t &\mapsto (f\circ f)(t) = \exp{\left(-e^{-t}\right)}.
    \end{align*}
    Temos \(\ln{g(t)} = -f(t)\) para todo \(t \in [0,\infty)\), portanto
    \begin{equation*}
        g'(t) = f(t) g(t),
    \end{equation*}
    é a derivada de \(g\), já que \(f'(t) = -f(t)\). Assim, a segunda derivada é dada por
    \begin{align*}
        g''(t) &= f'(t)g(t) + f(t)g'(t)\\
               &= -f(t) g(t) + f(t) f(t) g(t)\\
               &= \left(f(t) - 1\right) f(t) g(t).
    \end{align*}
    Notemos que
    \begin{align*}
        t \in [0, \infty) &\implies 0 < f(t) \leq 1\\
                          &\implies -1 < f(t) - 1 \leq 0
    \end{align*}
    onde a igualdade ocorre apenas para \(t = 0\), portanto \(g''(t) \leq 0\) para todo \(t \geq 0\), isto é, \(g'(t)\) decresce monotonicamente em \([0, \infty)\). Desse modo, temos
    \begin{equation*}
        t \in [0, \infty) \implies 0 < g'(t) \leq \frac1e.
    \end{equation*}

    Consideremos \(\abs{g(x) - g(y)}\) para \(x,y \in [0,\infty)\). Pelo teorema do valor médio, existe \(\xi \in [0, \infty)\) entre \(x\) e \(y\) tal que
    \begin{equation*}
        g(x) - g(y) = g'(\xi) (x-y)
    \end{equation*}
    então
    \begin{equation*}
        \abs{g(x) - g(y)} = g'(\xi)\abs{x - y} \leq \frac1e \abs{x - y}.
    \end{equation*}

    Assim, mostramos que \(g\) é uma contração em \([0, \infty)\) em relação à métrica usual. Como \([0, \infty) = [0,1]\cup[1, \infty)\), temos pelo \cref{lem:união_completa} e pelas \cref{prop:fechado,prop:intervalo1inf} que este intervalo, dotado da métrica usual, é um espaço métrico completo. Assim, segue que \(g\) tem um único ponto fixo neste espaço, pelo teorema do ponto fixo de Banach. Pelo \cref{lem:ponto_fixo_iteração}, segue que \(f\) tem um único ponto fixo, que é \(\Omega\).
\end{proof}

Utilizando o código a seguir, obtemos após quarenta iterações um erro absoluto compatível com a precisão máxima oferecida pela biblioteca \verb|numpy|. O valor obtido pelo programa foi \(\Omega \simeq0.567143290409783873\). Após quarenta iterações e utilizando o ponto inicial \(x_0 = 0,\) o erro estimado é \(\Omega - x_{40} \leq \frac{e^{-41}}{1 - e^{-1}}.\)
\begin{listing}[htbp]
    \inputminted[linenos]{Python}{exercício10.py}
\end{listing}
