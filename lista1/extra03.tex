\section*{Exercício extra 3}
\begin{definition}{Mapa logístico}{mapa_logístico}
    A aplicação
    \begin{align*}
        T_a : \mathbb{R} &\to \mathbb{R}\\
                       x &\mapsto ax(1 - x)
    \end{align*}
    é chamada de \emph{mapa logístico} ao parâmetro \(a\).
\end{definition}

\begin{proposition}{Pontos fixos do mapa logístico}{pontos_fixo_mapa_logístico}
    Os pontos fixos do mapa logístico \(T_a\) são dados por
    \begin{equation*}
        x^\alpha = 0 \quad\text{e}\quad x^\beta = \frac{a-1}{a}.
    \end{equation*}
    O ponto fixo \(x^\beta\) pertence a \([0,1]\) só se \(a \geq 1\).
\end{proposition}
\begin{proof}
    A equação de ponto fixo para \(T_a\) é dada por
    \begin{equation*}
        x = ax(1 - x) \implies x(a - 1 - ax) = 0,
    \end{equation*}
    cujas soluções são justamente \(x^\alpha\) e \(x^\beta\).

    Para \(a \geq 1\), temos
    \begin{equation*}
        a - 1 \geq 0 \implies \frac{a - 1}{a} \geq 0
    \end{equation*}
\end{proof}
