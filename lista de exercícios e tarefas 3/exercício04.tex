\section*{Exercício 4}
\begin{proposition}{Autofunções do problema de Sturm-Liouville com \(u'' + u' + \lambda u = 0\)}{exercício4d}
    Os autovalores e as autofunções normalizadas do problema de Sturm-Liouville \(u'' + u' + \lambda u = 0\) com condições de contorno \(u(0) = 0\) e \(u'(1) = 0\) são
    \begin{equation*}
        \lambda_n =\varphi_n^2+ \frac14
        \quad\text{e}\quad
        u_n(x) = \sqrt{\frac{1 + 4\varphi_n^2}{2\varphi_n^2}}\exp\left(-\frac{1}{2}x\right)\sin\left(\varphi_n x\right)
    \end{equation*}
    para \(n \in \mathbb{N}\smallsetminus{0}\), onde \(\varphi_n\) é a \(n\)-ésima raiz positiva da equação transcendental \(2\varphi = \tan\varphi\). Assim,
    \begin{equation*}
        G(x,\xi) =- 2\sum_{n = 1}^\infty \frac{\exp\left(-\frac{x + \xi}{2}\right)\sin\left(\varphi_n x\right)\sin\left(\varphi_n \xi\right)}{\varphi_n^2},
    \end{equation*}
    para \((x,\xi) \in [0,1]\times[0,1]\), é a função de Green do problema de Sturm associado.
\end{proposition}
\begin{proof}
    Como já feito na \cref{prop:exercício3b}, temos as soluções gerais da equação diferencial, que dependem se \(\lambda\) é nulo ou se \(\lambda - \frac14\) é positivo, nulo, ou negativo. Para \(\lambda = 0\), temos \(u(x) = \alpha e^{-x} + \beta\), portanto de \(u'(1) = 0\), segue que \(\alpha = 0\), levando à solução trivial. Para \(\lambda = \frac14\), temos \(u(x) = (\alpha x + \beta)e^{-\frac12 x}\), logo de \(u(0) = 0\), segue que \(\beta = 0\), portanto
    \begin{equation*}
        u'(x) = \alpha \left(1 - \frac12 x\right)e^{-\frac12 x},
    \end{equation*}
    e temos de \(u'(1) = 0\) que \(\alpha = 0\). Para \(\lambda < \frac14\) e \(\lambda \neq 0\), temos a solução geral
    \begin{equation*}
        u(x) = \alpha \exp\left(\lambda_+ x\right) + \beta \exp\left(\lambda_-x\right),
    \end{equation*}
    onde \(\lambda_+ = \frac{-1 + \sqrt{1 - 4 \lambda}}{2}\) e \(\lambda_- = \frac{-1 - \sqrt{1 - 4 \lambda}}{2}\) são valores reais e distintos. Das condições de contorno, temos
    \begin{equation*}
        \begin{cases}
            \alpha + \beta = 0\\
            \lambda_+ \alpha e^{\lambda_+} + \lambda_- \beta e^{\lambda_-} = 0
        \end{cases} \implies
        \alpha (\lambda_+ e^{\lambda_+} - \lambda_- e^{\lambda_-}) = 0,
    \end{equation*}
    e \todo[mostrando que \(\lambda_+ e^{\lambda_+} - \lambda_- e^{\lambda_-}\) não se anula], segue que há apenas a solução trivial. Para \(\lambda > \frac14\), temos a solução geral
    \begin{equation*}
        u(x) = e^{-\frac12x}\left[\alpha \cos\left(x\sqrt{\lambda - \frac14}\right) + \beta \sin\left(x\sqrt{\lambda - \frac14}\right)\right],
    \end{equation*}
    portanto da condição de contorno \(u(0) = 0\), segue que \(\alpha = 0\), e da condição de contorno \(u'(1) = 0\), segue que as soluções não triviais devem satisfazer
    \begin{equation*}
        2 \sqrt{\lambda - \frac14}\cos\left(\sqrt{\lambda - \frac14}\right) - \sin\left(\sqrt{\lambda - \frac14}\right) = 0 \implies 2\sqrt{\lambda - \frac14} = \tan\left(\sqrt{\lambda - \frac14}\right).
    \end{equation*}
    Notemos que a equação transcendental obtida possui infinitas soluções para \(\lambda > \frac14\), uma vez que a imagem da tangente contém todos os números reais positivos e a imagem da raiz quadrada também. Seja \(\varphi_n\) a \(n\)-ésima solução positiva da equação transcendental \(2\varphi = \tan \varphi\), então os autovalores do problema de Sturm-Liouville são dados por
    \begin{equation*}
        \lambda_n = \varphi_n^2 + \frac14,
    \end{equation*}
    para \(n \in \mathbb{N} \smallsetminus\set{0}\).

    Notemos que o produto interno para este problema de Sturm-Liouville é dado por
    \begin{equation*}
        \inner{f}{g}_r = \int_{0}^{1} e^{t}\dli{t}\overline{f(t)}g(t)
    \end{equation*}
    para quaisquer funções integráveis \(f, g\) em \([0,1]\). Para determinar as autofunções, devemos impor que as soluções encontradas têm norma unitária em relação a este produto interno, isto é,
    \begin{align*}
        \inner{u_n}{u_n}_r = 1 &\implies \int_0^1 e^t \dli{t} \abs{\beta_n}^2 e^{-t}\cos^2\left(\varphi_n t\right) = 1\\
                               &\implies  \frac12\abs{\beta_n}^2 \left[\int_0^1 \dli{t} - \int_0^1 \dli{t} \cos\left(2\varphi_n t\right)\right] = 1\\
                               &\implies \frac12 \left[1 - \frac{\sin(2\varphi_n)}{2\varphi_n}\right]\abs{\beta_n}^2 = 1.
    \end{align*}
    Da definição de \(\varphi_n\), temos que \(\tan\varphi_n > 0\), isto é, os sinais de \(\sin\varphi_n\) e de \(\cos\varphi_n\) são iguais, logo \(\sin(2\varphi_n) > 0\). Ainda, temos
    \begin{equation*}
        \sec^2\varphi_n = 1 + 4\varphi_n^2 \implies \cos^2\varphi_n = \frac{1}{1+4\varphi_n^2} \implies \sin^2\varphi_n = \frac{4\varphi_n^2}{1+4\varphi_n^2} \implies \sin(2\varphi_n) = \frac{2\varphi_n}{1+4\varphi_n^2},
    \end{equation*}
    portanto a constante de normalização é
    \begin{equation*}
        \frac12\left[1 - \frac{1}{1+4\varphi_n^2}\right]\abs{\beta_n}^2 = 1 \implies \beta_n = \sqrt{\frac{1+4\varphi_n^2}{2\varphi_n^2}}.
    \end{equation*}
    Deste modo,
    \begin{equation*}
        u_n(x) = \sqrt{\frac{1 + 4\varphi_n^2}{2\varphi_n^2}}\exp\left(-\frac{1}{2}x\right)\sin\left(\varphi_n x\right)
    \end{equation*}
    são as autofunções para este problema de Sturm-Liouville, para \(n \in \mathbb{N}\smallsetminus\set{0}\).

    Pela fórmula de Mercer, a função de Green do problema associado é
    \begin{align*}
        G(x,\xi) &= -\sum_{n = 1}^\infty \frac{u_n(x)u_n(\xi)}{\lambda_n}\\
                 &= - 2\sum_{n = 1}^\infty \frac{\exp\left(-\frac{x + \xi}{2}\right)\sin\left(\varphi_n x\right)\sin\left(\varphi_n \xi\right)}{\varphi_n^2},
    \end{align*}
    como desejado.
\end{proof}
