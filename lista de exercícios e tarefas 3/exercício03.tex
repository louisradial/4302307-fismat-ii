\section*{Exercício 3}
\begin{proposition}{Função de Green para o problema de Sturm \((e^x u'(x))' = f(x)\)}{exercício3a}
    Seja \(L\) o operador de Liouville dado por
    \begin{equation*}
        (Lu)(x) = \diff*{\left(e^x \diff{u}{x}\right)}{x}.
    \end{equation*}
    A função de Green para o problema de Sturm \(Lu = f\) com condições de contorno \(u(0) = u(1) = 0\) no intervalo \([0,1]\) é
    \begin{equation*}
        G(x,\xi) = \left\{\begin{aligned}
                \frac{(e^{-x} - 1)(e^{-\xi} - e^{-1})}{1 - e^{-1}}, && 0 \leq x \leq \xi \leq 1\\
                \frac{(e^{-\xi} - 1)(e^{-x} - e^{-1})}{1 - e^{-1}}, && 0 \leq \xi \leq x \leq 1
        \end{aligned}\right.
    \end{equation*}
    para \((x, \xi) \in [0,1]\times [0,1]\).
\end{proposition}
\begin{proof}
    Notemos que o núcleo do operador é dado por
    \begin{align}
        Lu = 0 &\implies u''(x) + u'(x) = 0\\
               &\implies u(x) = \alpha e^{-x} + \beta,
    \end{align}
    com \(\alpha, \beta\) constantes. Desse modo, a solução para os problemas
    \begin{equation*}
        \begin{cases}
            Lv_1 = 0\\
            v_1(0) = 0
        \end{cases}
        \quad\text{e}\quad
        \begin{cases}
            Lv_2 = 0\\
            v_2(1) = 0
        \end{cases}
    \end{equation*}
    são dadas por
    \begin{equation*}
        v_1(x) = A(e^{-x} - 1)\quad\text{e}\quad v_2(x) = B(e^{-x} - e^{-1}),
    \end{equation*}
    para constantes \(A, B\) não nulas. Assim, o determinante Wronskiano desse par de funções é
    \begin{align*}
        W(x) &= v_1(x) v_2'(x) - v_1'(x) v_2(x)\\
             &= AB e^{-x}(e^{-x} - e^{-1}) -AB e^{-x} (e^{-x} - 1)\\
             &= AB e^{-x} (1 - e^{-1}).
    \end{align*}
    Por fim, a função de Green é dada por
    \begin{align*}
        G(x,\xi) = \left\{\begin{aligned}
                \frac{v_1(x)v_2(\xi)}{e^{x}W(x)}, && 0 \leq x \leq \xi \leq 1\\
                \frac{v_1(\xi)v_2(x)}{e^{x}W(x)}, && 0 \leq \xi \leq x \leq 1
        \end{aligned}\right. &\implies G(x,\xi) = \left\{\begin{aligned}
                \frac{A(e^{-x} - 1)B(e^{-\xi} - e^{-1})}{AB(1 - e^{-1})}, && 0 \leq x \leq \xi \leq 1\\
                \frac{A(e^{-\xi} -1)B(e^{-x} - e^{-1})}{AB(1 - e^{-1})}, && 0 \leq \xi \leq x \leq 1
        \end{aligned}\right.
\\
        &\implies G(x,\xi) = \left\{\begin{aligned}
                \frac{(e^{-x} - 1)(e^{-\xi} - e^{-1})}{1 - e^{-1}}, && 0 \leq x \leq \xi \leq 1\\
                \frac{(e^{-\xi} - 1)(e^{-x} - e^{-1})}{1 - e^{-1}}, && 0 \leq \xi \leq x \leq 1
        \end{aligned}\right.
    \end{align*}
    como desejado.
\end{proof}

\begin{proposition}{Autofunções do problema de Sturm-Liouville \((e^xu'(x))' + \lambda e^x u(x) = 0\)}{exercício3b}
    Seja \(L\) o mesmo operador de Liouville definido na \cref{prop:exercício3a}. Os autovalores e as autofunções normalizadas do problema de Sturm-Liouville \((Lu)(x) + \lambda e^x u(x) = 0\) com condições de contorno \(u(0) = u(1) = 0\) no intervalo \([0,1]\) são
    \begin{equation*}
        \lambda_n = n^2 \pi^2 + \frac14
        \quad\text{e}\quad
        u_n(x) = \sqrt{2} \exp\left(-\frac{1}{2}x\right)\sin(n\pi x)
    \end{equation*}
    para todo \(n \in \mathbb{N}\smallsetminus\set{0}\).
\end{proposition}
\begin{proof}
    Notemos que
    \begin{equation*}
        (Lu)(x) + \lambda e^{x} u(x) = e^x \left(\diff[2]{u}{x} + \diff{u}{x} + \lambda u(x)\right),
    \end{equation*}
    logo as soluções do problema de Sturm-Liouville são soluções da equação diferencial ordinária
    \begin{equation*}
        \diff[2]{u}{x} + \diff{u}{x} + \lambda u(x) = 0,
    \end{equation*}
    com condições de contorno \(u(0) = u(1) = 0\), já que \(e^x \neq 0\).

    Para \(\lambda = 0\), já vimos na \cref{prop:exercício3a} que a solução geral é
    \begin{equation*}
        u(x) = \alpha e^{-x} + \beta.
    \end{equation*}
    Das condições de contorno temos
    \begin{equation*}
        \begin{cases}
            \alpha + \beta = 0\\
            \alpha e^{-1} + \beta = 0
        \end{cases}\implies \alpha = \beta = 0,
    \end{equation*}
    isto é, o caso \(\lambda = 0\) admite apenas a solução trivial. Para \(\lambda < \frac14\) e \(\lambda \neq 0\), temos a solução geral
    \begin{equation*}
        u(x) = \alpha \exp\left(\frac{-1 + \sqrt{1 - 4 \lambda}}{2}x\right) + \beta \exp\left(\frac{-1 - \sqrt{1 - 4 \lambda}}{2}x\right),
    \end{equation*}
    portanto das condições de contorno temos
    \begin{equation*}
        \begin{cases}
            \alpha + \beta = 0\\
        \alpha \exp\left(\frac{-1 + \sqrt{1 - 4 \lambda}}{2}\right) + \beta \exp\left(\frac{-1 - \sqrt{1 - 4 \lambda}}{2}\right) = 0
        \end{cases}\implies \alpha = \beta = 0,
    \end{equation*}
    isto é, este caso também admite apenas a solução trivial. Para \(\lambda = \frac14\), temos a solução geral
    \begin{equation*}
        u(x) = (\alpha x + \beta)e^{- \frac12 x}
    \end{equation*}
    e segue trivialmente que apenas a solução trivial satisfaz as condições de contorno. Por fim,para \(\lambda > \frac14\), temos a solução geral
    \begin{equation*}
        u(x) = e^{-\frac12x}\left[\alpha \cos\left(x\sqrt{\lambda - \frac14}\right) + \beta \sin\left(x\sqrt{\lambda - \frac14}\right)\right],
    \end{equation*}
    portanto da condição de contorno \(u(0) = 0\), segue que \(\alpha = 0\), e da condição de contorno \(u(1) = 0\), segue que as soluções não triviais devem satisfazer
    \begin{equation*}
        \sin\left(\sqrt{\lambda - \frac14}\right) = 0 \implies \sqrt{\lambda - \frac14} = n\pi,
    \end{equation*}
    com \(n \in \mathbb{N}\smallsetminus\set{0}.\) Isto é, os autovalores para este problema de Sturm-Liouville são
    \begin{equation*}
        \lambda_n = n^2 \pi^2 + \frac14,
    \end{equation*}
    para \(n \in \mathbb{N} \smallsetminus\set{0}\).

    Notemos que o produto interno para este problema de Sturm-Liouville é dado por
    \begin{equation*}
        \inner{f}{g}_r = \int_{0}^{1} e^{t}\dli{t}\overline{f(t)}g(t)
    \end{equation*}
    para quaisquer funções integráveis \(f, g\) em \([0,1]\). Para determinar as autofunções, devemos impor que as soluções encontradas têm norma unitária em relação a este produto interno, isto é,
    \begin{align*}
        \inner{u_n}{u_n}_r = 1 &\implies \int_0^1 e^t \dli{t} \abs{\beta_n}^2 e^{-t}\sin^2\left(n\pi t\right) = 1\\
                               &\implies  \frac12\abs{\beta_n}^2 \left[\int_0^1 \dli{t} - \int_0^1 \dli{t} \cos(2n\pi t)\right] = 1\\
                               &\implies  \abs{\beta_n}^2 = 2.
    \end{align*}
    Deste modo,
    \begin{equation*}
        u_n(x) = \sqrt{2} \exp\left(-\frac{1}{2}x\right)\sin(n\pi x)
    \end{equation*}
    são as autofunções para este problema de Sturm-Liouville, para \(n \in \mathbb{N}\smallsetminus\set{0}\).
\end{proof}
\begin{corollary}
    A função de Green dada pela \cref{prop:exercício3a} pode ser dada pela série
    \begin{equation*}
        G(x, \xi) = - 8\sum_{n = 1}^\infty \frac{\exp\left(-\frac{x+\xi}2\right)\sin(n\pi x)\sin(n\pi \xi)}{1 + 4n^2\pi^2}.
    \end{equation*}
\end{corollary}

\begin{lemma}{Aplicação do problema de Sturm-Liouville}{sec1933}
    Sejam \(\lambda_n \in \mathbb{R} \smallsetminus\set{0}\) e \(u_n : \mathbb{R} \to \mathbb{R}\) os autovalores e as autofunções normalizadas do problema de Sturm-Liouville regular \(Lu + \lambda ru = 0\), com condições de contorno lineares e homogêneas no intervalo \([a,b]\subset \mathbb{R}\). Seja \(f : \mathbb{R} \to \mathbb{R}\) uma função contínua e seja \(\gamma \in \mathbb{R}\) um número real que não seja um autovalor do problema de Sturm-Liouville, então a solução da equação diferencial
    \begin{equation*}
        Lu + \gamma ru = f
    \end{equation*}
    sujeita às mesmas condições de contorno que o problema de Sturm-Liouville é dada por
    \begin{equation*}
        u(x) = \sum_{\ell = 1}^\infty \frac{\inner{u_\ell}{f}}{\gamma - \lambda_\ell}u_\ell(x).
    \end{equation*}
\end{lemma}
\begin{proof}
    Seja \(G : [a,b]\times[a,b] \to \mathbb{R}\) a função de Green para o problema de Sturm associado. Então, temos
    \begin{equation*}
        Lu = f - \gamma r u \implies u(x) = \underbrace{\int_a^b \dli{\xi} G(x, \xi) f(\xi)}_{g(x)} - \gamma \int_a^b \dli{\xi} r(\xi)G(x,\xi) u(\xi).
    \end{equation*}
    Pela fórmula de Mercer, a função de Green é dada pela série uniformemente convergente
    \begin{equation*}
        G(x,\xi) = -\sum_{n = 1}^\infty \frac{u_n(x)u_n(\xi)}{\lambda_n},
    \end{equation*}
    portanto podemos escrever
    \begin{align*}
        u(x) &= g(x) + \gamma \int_{a}^{b} r(\xi) \dli{\xi} \sum_{n = 1}^\infty \frac{u_n(x)u_n(\xi)}{\lambda_n} u(\xi)\\
             &= g(x) + \gamma \sum_{n=1}^\infty \frac{u_n(x)}{\lambda_n} \int_{a}^{b} r(\xi)\dli{\xi} u_n(\xi)u(\xi)\\
             &= g(x) + \gamma \sum_{n=1}^\infty \frac{\inner{u_n}{u}_r}{\lambda_n}u_n(x).
    \end{align*}

    Tomando o produto interno com \(u_m\), obtemos
    \begin{equation*}
        \inner{u_m}{u}_r = \inner{u_m}{g}_r + \gamma \sum_{n = 1}^\infty \frac{\inner{u_n}{u}_r}{\lambda_n} \inner{u_m}{u_n}_r,
    \end{equation*}
    portanto, como \(\inner{u_m}{u_n}_r = \delta_{mn}\), segue que
    \begin{equation*}
        \inner{u_m}{u}_r = \frac{\lambda_m \inner{u_m}{g}_r}{\lambda_m - \gamma}.
    \end{equation*}
    Substituindo na expressão para \(u\), temos
    \begin{equation*}
        u(x) = g(x) + \gamma \sum_{n = 1}^\infty \frac{\inner{u_n}{g}_r}{\lambda_n - \gamma}u_n(x).
    \end{equation*}

    Utilizando a fórmula de Mercer mais uma vez, obtemos
    \begin{align*}
        g(x) &= -\int_a^b \dli{\xi} \sum_{k = 1}^\infty \frac{u_k(x)u_k(\xi)}{\lambda_k} f(\xi)\\
             &= - \sum_{k = 1}^\infty \frac{u_k(x)}{\lambda_k} \int_{a}^{b}\dli{\xi} u_k(\xi) f(\xi)\\
             &= - \sum_{k = 1}^\infty \frac{\inner{u_k}{f}}{\lambda_k}u_k(x).
    \end{align*}
    Tomando o produto interno com \(u_n\), temos
    \begin{equation*}
        \inner{u_n}{g}_r = -\frac{\inner{u_n}{f}}{\lambda_n}.
    \end{equation*}
    Substituindo na expressão para \(u\), obtemos
    \begin{align*}
        u(x) &= - \sum_{k=1}^\infty \frac{\inner{u_k}{f}}{\lambda_k}u_k(x) + \sum_{n=1}^\infty\frac{\gamma\inner{u_n}{f}}{\lambda_n(\gamma - \lambda_n)}u_n(x)\\
             &= \sum_{\ell = 1}^\infty \left(\frac{\gamma}{\gamma - \lambda_\ell} - 1\right)\frac{\inner{u_\ell}{f}}{\lambda_\ell}u_\ell(x)\\
             &= \sum_{\ell = 1}^\infty \frac{\inner{u_\ell}{f}}{\gamma - \lambda_\ell}u_\ell(x),
    \end{align*}
    como queríamos mostrar.
\end{proof}
\begin{corollary}
    A solução da equação diferencial
    \begin{equation*}
        (e^xu'(x))' + 5e^x u(x) = f(x)
    \end{equation*}
    para \(f(x) = e^{\frac{x}{2}}\), com condições de contorno \(u(0) = u(1) = 0\) no intervalo \([0,1]\) é
\end{corollary}
\begin{proof}
    Uma vez que 5 não é autovalor do problema de Sturm-Liouville associado, precisamos apenas determinar os produtos internos \(\inner{u_\ell}{f}\), segundo o \cref{lem:sec1933}.

    Temos
    \begin{equation*}
        \inner{u_\ell}{f} = \int_0^1 \dli{\xi} \sqrt{2} \sin(\ell\pi x) = \frac{\sqrt{2}\left[1 - (-1)^\ell\right]}{\ell \pi}
    \end{equation*}
    para todo \(\ell \in \mathbb{N} \smallsetminus\set{0}\). Portanto, \(\inner{u_{2\ell}}{f} = 0\) e
    \begin{equation*}
        \inner{u_{2\ell-1}}{f} = \frac{2\sqrt{2}}{(2\ell - 1)\pi}.
    \end{equation*}
    Assim, a solução da equação diferencial é
    \begin{align*}
        u(x) &= \frac{4}{\pi} \sum_{\ell = 1}^\infty \frac{\exp\left(-\frac12x\right)\sin\left[(2\ell - 1)\pi x\right]}{[5 - \frac{1}{4} - (2\ell - 1)^2\pi^2](2\ell - 1)}\\
             &= \frac{16}{\pi} \sum_{\ell = 1}^\infty \frac{\exp\left(-\frac12x\right)\sin\left[(2\ell - 1)\pi x\right]}{[19 - (4\ell - 2)^2\pi^2](2\ell - 1)},
    \end{align*}
    como desejado.
\end{proof}
