\section*{Exercício 2}
\begin{corollary}\label{exercício2a}
    A função de Green do problema de Sturm \(u''(x) = f(x)\) onde \(u\) é definida no intervalo \(x \in [0,1]\) e satisfaz \(u'(0) = 0\) e \(u(1) = 0\) é dada por
    \begin{equation*}
        G(x,\xi) = \left\{\begin{aligned}
                \xi - 1, && 0 \leq x < \xi \leq 1\\
                x - 1, && 0 \leq \xi < x \leq 1
        \end{aligned}\right.
    \end{equation*}
\end{corollary}
\begin{proof}
    Identificando \(\alpha_1 = 0, \alpha_2 = 1, \beta_1 = 1,\) e \(\beta_2 = 0,\) o resultado segue da \cref{prop:exercício1}.
\end{proof}

\begin{proposition}{Autovalores e autofunções do problema de Sturm-Liouville \(u'' + \lambda u = 0\)}{exercício2b}
    Os autovalores e as autofunções normalizadas do problema de Sturm-Liouville
    \begin{equation*}
        u'' + \lambda u = 0,
    \end{equation*}
    onde \(u\) é definida no intervalo \(x \in [0,1]\) e satisfaz as condições de contorno \(u'(0) = 0\) e \(u(1) = 0\) são dados por
    \begin{equation*}
        \lambda_n = \left(n-\frac{1}{2}\right)^2\pi^2\quad\text{e}\quad
        u_n =\sqrt{2} \cos\left(\frac{2n-1}{2}\pi x\right),
    \end{equation*}
    para todo \(n \in \mathbb{N}\smallsetminus\set{0}.\)
\end{proposition}
\begin{proof}
    Notamos que a solução geral da equação diferencial do problema de Sturm-Liouville é
    \begin{equation*}
        u(x) = \begin{cases}
            A\cosh(\sqrt{-\lambda}x) + B\sinh(\sqrt{-\lambda}x), & \lambda < 0\\
            Ax + B, &\lambda = 0\\
            A\cos(\sqrt{\lambda}x)+ B\sin(\sqrt{\lambda}x),&\lambda > 0\\
        \end{cases}.
    \end{equation*}
    Notemos que pelas condições de contorno, a solução para \(\lambda = 0\) é a solução trivial, portanto podemos descartar este caso. Para \(\lambda < 0\), segue de \(u'(0) = 0\) que \(B = 0\), então como o cosseno hiperbólico tem imagem positiva, a única solução de \(u(1) = 0\) é \(A = 0\), isto é, este caso também leva apenas a soluções triviais. Nos resta apenas o caso \(\lambda > 0\), temos de \(u'(0) = 0\) que \(B = 0\), logo da outra condição de contorno obtemos
    \begin{equation*}
        u(1) = 0 \implies \sqrt{\lambda} = \left(n-\frac12\right)\pi,\quad\text{com}\quad n \in \mathbb{N}\smallsetminus\set{0}.
    \end{equation*}
    Deste modo, os autovalores do problema de Sturm-Liouville considerado são
    \begin{equation*}
        \lambda_n = \left(n-\frac12\right)^2\pi^2,
    \end{equation*}
    para \(n \in \mathbb{N}\smallsetminus\set{0}\).

    Para determinar as autofunções normalizadas, notamos que o produto interno para este problema de Sturm-Liouville coincide com o produto interno usual para o espaço de funções integráveis em \([0,1]\). Impondo que \(\inner{u_n}{u_n} = 1\), obtemos
    \begin{equation*}
        \int_{0}^{1} \dli{x}\abs{A}^2 \cos^2\left(\frac{2n-1}{2}\pi x\right) = 1 \implies \abs{A}^2 \int_{0}^{1}\dli{x} \frac{1 + \cos((2n-1)\pi x)}{2} = 1 \implies \abs{A} = \sqrt{2},
    \end{equation*}
    portanto as autofunções normalizadas do problema de Sturm-Liouville são
    \begin{equation*}
        u_n(x) = \sqrt{2} \cos\left(\frac{2n-1}{2}\pi x\right),
    \end{equation*}
    para \(n \in \mathbb{N}\smallsetminus\set{0}.\)
\end{proof}

\begin{corollary}\label{exercício2c}
    A função de Green para o problema de Sturm associado é dada por
    \begin{equation*}
        G(x, \xi) = - \frac{8}{\pi^2}\sum_{m = 0}^{\infty} \frac{\cos\left(\frac{2m + 1}{2}\pi x\right)\cos\left(\frac{2m + 1}{2}\pi \xi\right)}{(2m + 1)^2},
    \end{equation*}
    para todo \((x, \xi) \in [0,1]\times[0,1]\).
\end{corollary}
\begin{proof}
    Pela fórmula de Mercer, temos
    \begin{equation*}
        G(x,\xi) = - \sum_{n = 1}^{\infty} \frac{u_n(x) u_n(\xi)}{\lambda_n},
    \end{equation*}
    onde \(u_n\) é a autofunção normalizada associada ao autovalor \(\lambda_n\) do problema de Sturm-Liouville. Pela \cref{prop:exercício2b}, segue que
    \begin{align*}
        G(x, \xi) &= - \sum_{n = 1}^\infty \frac{\left[\sqrt{2}\cos\left(\frac{2n-1}{2}\pi x\right)\right]\left[\sqrt{2}\cos\left(\frac{2n-1}{2}\pi \xi\right)\right]}{\left(\frac{2n-1}{2}\right)^2\pi^2}\\
                  &= - \frac{8}{\pi^2} \sum_{n = 1}^\infty \frac{\cos\left(\frac{2n-1}{2}\pi x\right)\cos\left(\frac{2n-1}{2}\pi \xi\right)}{(2n-1)^2}.
    \end{align*}
    Fazendo a troca de variável de soma \(m = n - 1\), obtemos a expressão desejada.
\end{proof}
\begin{corollary}
    Vale a identidade
    \begin{equation*}
        \frac{\pi^2}{8} = \sum_{m = 0}^\infty \frac{1}{(2m+1)^2}.
    \end{equation*}
\end{corollary}
\begin{proof}
    Pelo \cref{exercício2a} e pela continuidade da função de Green, segue que \(G(0,0) = -1\). Desse modo, pelo \cref{exercício2c}, temos
    \begin{equation*}
        -\frac{8}{\pi^2} \sum_{m=0}^\infty \frac{1}{(2m+1)^2} = -1\implies
        \sum_{m=0}^\infty \frac{1}{(2m+1)^2} = \frac{\pi^2}{8},
    \end{equation*}
    como desejado.
\end{proof}

\begin{proposition}{Solução para o problema de Sturm \(u''(x) = (3 - x)e^x\)}{exercício2d}
    A solução do problema de Sturm
    \begin{equation*}
        u''(x) = (3-x)e^x
    \end{equation*}
    com condições de contorno \(u'(0) = 0\) e \(u(1) = 0\) é
    \begin{equation*}
        u(x) = (5-x)e^x - 4(x - 1 + e)
    \end{equation*}
    para \(x \in [0,1]\).
\end{proposition}
\begin{proof}
    Utilizando a função de Green obtida no \cref{exercício2a}, a solução deste problema de Sturm é dada por
    \begin{align*}
        u(x) &= \int_{0}^{1}\dli\xi G(x,\xi) (3-\xi)e^\xi\\
             &= (x - 1)\int_0^x \dli\xi (3-\xi)e^\xi + \int_x^1 \dli\xi (\xi - 1)(3-\xi)e^{\xi}.
    \end{align*}
    Integrando por partes, obtemos
    \begin{equation*}
        \int_0^x \dli\xi (3-\xi)e^\xi = (4 - x)e^x - 4\quad\text{e}\quad \int_x^1 \dli\xi(\xi - 1)(3 -\xi)e^\xi = (x - 3)^2e^x - 4e,
    \end{equation*}
    logo a solução do problema de Sturm é
    \begin{align*}
        u(x) &= (x - 1)(4-x)e^x - 4(x-1) + (x - 3)^2e^x - 4e\\
             &= (5 - x)e^x - 4(x - 1 + e).
    \end{align*}

    Verificamos que é de fato solução da equação diferencial calculando as duas primeiras derivadas desta função, obtendo
    \begin{equation*}
        u'(x) = (4 - x)e^x - 4\quad\text{e}\quad u''(x) = (3 - x)e^x.
    \end{equation*}
    Assim, como \(u'(0) = 0\) e \(u(1) = 0\), esta é de fato a solução do problema de Sturm.
\end{proof}
